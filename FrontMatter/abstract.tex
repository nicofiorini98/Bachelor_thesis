\begingroup 
\titleformat{\chapter}
{\normalfont\Huge\bfseries\centering} %shape
%{\centering} % format
{}% label
{} % sep
{}  

\chapter*{Abstract}


L' industria del software si trova a fronteggiare la 
necessità di sviluppare software sempre più scalabili 
e performanti per fronteggiare l'aumento degli utenti 
e di servizi che ne fanno utilizzo.
In questo contesto, Elixir, un linguaggio di programmazione 
funzionale e concorrente basato su Erlang, emerge come una 
scelta promettente per la costruzione di sistemi altamente 
affidabili e reattivi, semplificando di molto lo sviluppo 
di software concorrenziale.

Questo studio si propone di analizzare le caratteristiche 
di Elixir e le sue performance attraverso una serie di 
esperimenti empirici esplorando diversi aspetti delle performance 
mettendo in rilievo vantaggi e svantaggi nell'adottarlo.

I risultati di questa ricerca forniranno una comprensione 
approfondità delle capacità di Elixir in termini di prestazioni
e affidabilità consentendo agli sviluppatori di fare una scelta
pensata alle esigenze dei loro progetti.

\endgroup