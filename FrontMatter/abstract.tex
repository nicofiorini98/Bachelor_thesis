\begingroup 
\titleformat{\chapter}
{\normalfont\Huge\bfseries\centering} %shape
%{\centering} % format
{}% label
{} % sep
{}  

\chapter*{Abstract}

L' industria del software si trova a fronteggiare la 
necessità di sviluppare software sempre più scalabili 
e performanti per fronteggiare l'aumento degli utenti 
e di servizi che ne fanno utilizzo.
In questo contesto, Elixir, un linguaggio di programmazione 
funzionale e concorrente basato su Erlang, emerge come una 
scelta promettente per la costruzione di sistemi altamente 
affidabili e reattivi, nonché la sua capacità di
gestire carichi di lavoro intensivi e flussi di dati in
tempo reale tipici delle applicazioni IoT.

Questo studio si propone di analizzare le caratteristiche 
di Elixir che lo rendono un linguaggio degno di nota,
e attraverso una serie di esperimenti empirici si misurano le
performance di Elixir, anche mettendolo a confronto con altre
soluzioni più diffuse.

I risultati di questa ricerca forniscono una base per valutare
l'efficacia di Elixir, contribuendo così alla comprensione del
ruolo che questo linguaggio può svolgere nel futuro dello
sviluppo delle applicazioni IoT.

\endgroup