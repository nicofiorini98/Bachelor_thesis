\begingroup 
\titleformat{\chapter}
{\normalfont\Huge\bfseries\centering} %shape
%{\centering} % format
{}% label
{} % sep
{}  

\chapter*{Abstract}

L'industria del software è attualmente alle prese con
la crescente esigenza di sviluppare applicazioni
sempre più scalabili e performanti per gestire efficacemente
l'aumento del numero di utenti e dei servizi
che utilizzano tali applicazioni.
In particolare, nel contesto dell'Internet of Things (IoT),
un settore in costante sviluppo, la reattività,
la tolleranza agli errori e la scalabilità emergono
come fattori cruciali.

In questo contesto, Elixir, un linguaggio di programmazione 
funzionale e concorrente basato su Erlang, emerge come una 
scelta promettente per la costruzione di sistemi altamente 
affidabili e reattivi, nonché la sua capacità di
gestire carichi di lavoro intensivi e flussi di dati in
tempo reale tipici delle applicazioni IoT.

Questo studio si propone di analizzare le caratteristiche 
di Elixir che lo rendono un linguaggio degno di nota,
e attraverso una serie di esperimenti empirici si misurano le
performance di Elixir, anche mettendolo a confronto con altre
soluzioni più diffuse.

I risultati di questa ricerca forniscono una base per valutare
l'efficacia di Elixir, contribuendo così alla comprensione del
ruolo che questo linguaggio può svolgere nel futuro dello
sviluppo delle applicazioni IoT.

\endgroup