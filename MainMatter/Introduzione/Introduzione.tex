\chapter{Introduzione}

Elixir è un linguaggio sviluppato nel 2012 da José Valim,
con l'obbiettivo di consentire una maggiore estensibilità
e produttività nella macchinia virtuale di Erlang,
mantenendo al contempo la compatibilità con l'ecosistema
di Erlang\cite{Elixirpr12:online}.
Per questo Elixir si è affermato come una scelta promettente
nell'industria del software, mettendosi in rilievo proprio
dove l'applicazione ha bisogno di essere scalabile, tollerante
agli errori, e responsivo grazie all' approccio concorrenziale.

In particolare, Elixir può essere una scelta utile anche nel
dominio dell' IoT per vari motivi: 
\begin{itemize}
	\item \textbf{Concorrenza}: Nell'IoT la concorrenza
	e la scalabilità sono fondamentali, proprio in quest'ambito
	possono esserci numerosi dispositivi che devono essere monitorati
	e controllati contemporaneamente,e la capacità di gestire
	facilmente la concorrenza è fondamentale.
	%TODO migliorare fault tolerance
	\item \textbf{Fault Tolerance}: Soprattutto controllando
	dispositivi hardware, dobbiamo tenere in conto che questi
	possono improvvisamente smettere di funzionare, ed Elixir
	può essere in grado di controllare processi che vanno in
	errore facilmente con la supervisione. 
	\item \textbf{Sviluppo Rapido e Manutenzione}: Elixir è un linguaggio
	abbastanza recente, la sintassi è snella e con feature essenziali,
	fornendo un build tool (Mix) che fornisce la gestione delle dipendenze
	attraverso il package Manager Hex \cite{Hex63:online},
	un ambiente interattivo(iex) e supporta la generazione automatica della
	documentazione .
\end{itemize}


Per questo Elixir si è affermato come una scelta promettente
nell'industria del software, proprio nell'a
Elixir si è affermato come una scelta promettente
nell'industria del software grazie ai vantaggi offerti
dal paradigma funzionale e dall'approccio 
all'immutabilità dei dati.
Questa combinazione consente lo sviluppo di software
estremamente robusto. 
In particolare, l'uso dell'immutabilità dei dati in 
Elixir semplifica significativamente la programmazione
concorrente eliminando la necessità di gestire dati
condivisi tra processi, il che porta a una maggiore
chiarezza e affidabilità nel codice evitando i possibili
comportamenti indeterminati che si potrebbero ottenere
in un linguaggio basato su stati.


Il codice è compilato per la macchina virtuale Beam,
anche detta Erlang VM, quindi un programmatore Elixir 
può chiamare qualsiasi funzione Erlang senza 
un costo di overhead durante l'esecuzione,
dando ampio spazio alla completa piattaforma OTP di
cui Erlang fa uso.\cite{OpenTele88:online}

Elixir in particolare è un linguaggio moderno, infatti
arriva con molti vantaggi nell'utilizzarlo rispetto ad
Erlang, infatti tutto ciò che si può fare con Elixir
si può fare anche con Erlang, ma Elixir porta ad un
livello superiore la semplicità di espressione
del codice attraverso una sintassi snella e comprensiva.



