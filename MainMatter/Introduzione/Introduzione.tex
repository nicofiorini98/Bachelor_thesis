\chapter{Introduzione}

Elixir è un linguaggio di programmazione dinamico e funzionale
sviluppato nel 2012 da José Valim,
con l'obbiettivo di favorire una maggiore scalabilità
e produttività nella macchinia virtuale di Erlang,
mantenendo al contempo la compatibilità con l'ecosistema
di Erlang\cite{Elixirpr12:online}.
Elixir si è affermato come una promettente scelta
nell'industria del software,specialmente in contesti dove è
richiesta scalabilità, tolleranza agli errori e reattività 
grazie al suo approccio concorrenziale.


In particolare, Elixir può risultare vantaggioso nel campo dell'IoT per diversi motivi:
\begin{enumerate}
	\item \textbf{Concorrenza}: Nell'ambito dell'IoT, la gestione simultanea
	di dispositivi è essenziale. Elixir, grazie alla sua capacità 
	di gestire facilmente la concorrenza, consente il monitoraggio
	e il controllo efficiente di numerosi dispositivi contemporaneamente.
	\item \textbf{Fault Tolerance}: 
	Data la natura degli ambienti IoT, dove i dispositivi possono
	guastarsi improvvisamente, Elixir offre strumenti per la supervisione
	e la gestione degli errori, garantendo la continuità delle operazioni
	anche in caso di fallimenti.
	\item \textbf{Sviluppo Rapido e Manutenzione}:
	Elixir è un linguaggio moderno che offre una sintassi efficiente e snella,
	oltre a strumenti di sviluppo come Mix per la gestione delle dipendenze
	e l'ambiente interattivo iex. La presenza di un package manager (Hex)\cite{Hex63:online}
	e la possibilità di generare automaticamente la documentazione
	facilitano il processo di sviluppo e manutenzione del codice.
	
\end{enumerate}

Il trattato esplora Elixir concentrandosi su due aspetti principali:
la semplicità e le performance. Si analizzano i punti di forza di 
un linguaggio funzionale e come questi sono sfruttati in Elixir, con 
un focus sulla concorrenza. Nella scelta di un linguaggio, la semplicità
è fondamentale e deve essere accessibile a tutti i programmatori.
Tuttavia, l'efficienza è altrettanto importante, quindi vengono condotti
test empirici per valutare le performance di Elixir.

In particolare il lavoro effettuato è così ripartito: 


\begin{itemize}
	\item Nel capitolo 2 si discute del linguaggio funzionale,
	esaminando le astrazioni offerte da Elixir per lo svilluppo
	di codice affidabile, si tratta la concorrenza e come la Erlang VM
	si occupa della gestione dei processi.
	% TODO migliorare
	\item Nel capitolo 3 si spiega il lavoro sperimentale svolto e i risultati
	ottenuti (continuare)

\end{itemize}


