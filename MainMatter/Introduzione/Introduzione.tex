\chapter{Introduzione}

Elixir è un linguaggio di programmazione dinamico e funzionale
sviluppato nel 2012 da José Valim,
con l'obbiettivo di consentire una maggiore estensibilità
e produttività nella macchinia virtuale di Erlang,
mantenendo al contempo la compatibilità con l'ecosistema
di Erlang\cite{Elixirpr12:online}.
Per questo Elixir si è affermato come una scelta promettente
nell'industria del software, mettendosi in rilievo proprio
dove l'applicazione ha bisogno di essere scalabile, tollerante
agli errori, e responsivo grazie all' approccio concorrenziale.

In particolare, Elixir può essere una scelta utile anche nel
dominio dell' IoT per vari motivi, alcuni dei quali sono: 
\begin{enumerate}
	\item \textbf{Concorrenza}: Nell'IoT la concorrenza
	e la scalabilità sono fondamentali, proprio in quest'ambito
	possono esserci numerosi dispositivi che devono essere monitorati
	e controllati contemporaneamente,e la capacità di gestire
	facilmente la concorrenza è fondamentale.
	%TODO migliorare fault tolerance
	\item \textbf{Fault Tolerance}: Soprattutto controllando
	dispositivi hardware, dobbiamo tenere in conto che questi
	possono improvvisamente smettere di funzionare, ed Elixir
	può essere in grado di controllare processi che vanno in
	errore facilmente con la supervisione. 
	\item \textbf{Sviluppo Rapido e Manutenzione}: Elixir è un linguaggio
	moderno, imparando dalla storia ormai i linguaggi moderni forniscono
	una sintassi efficiente e snella con feature essenziali,
	fornendo un build tool (Mix) che fornisce la gestione delle dipendenze
	attraverso il package Manager Hex \cite{Hex63:online},
	un ambiente interattivo(iex) e supporta la generazione automatica della
	documentazione.
\end{enumerate}

Il lavoro di tesi vuole mettere in luce i punti di forza di un linguaggio
funzionale e come essi vengono sfruttati in Elixir, si vuole mettere in
evidenza anche la semplicità nell'usare più processi, nel fare una scelta
la semplicità anche ricopre un ruolo
chiave, deve essere adatto e alla portata di
qualsiasi programmatore. La semplicità è un punto chiave ma un buon linguaggio
deve essere il più efficiente possibile almeno per gli ambiti di interesse,
per questo vengono fatti una serie di test empirici per valutare le performance
di questo.

tutto sulle performance probabilmente non esisterebbe python.


In particolare il lavoro effettuato è così ripartito: 

\begin{itemize}
	% \item Da correggere
	\item Nel capitolo 2 si accenna al linguaggio funzionale ed al
	lambda calcolo, come questo paradigma differisce
	rispetto ad un linguaggio basato su stati. Si spiegano anche
	le astrazioni che Elixir fornisce per poter	sviluppare codice affidabile
	\item Nel capitolo 3 si spiega il lavoro sperimentale svolto e i risultati
	ottenuti

\end{itemize}


