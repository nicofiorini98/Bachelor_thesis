\section{Il paradigma funzionale}
\renewcommand\lstlistingname{Esempio}
Come già accennato Elixir è un linguaggio funzionale, dove
il concetto di funzione ricopre il ruolo di protagonista,
i dati sono immutabili e il codice è dichiarativo.

Questo modo di vedere le cose deriva dal Lambda calcolo o $\lambda$-calcolo
\cite{Lambdaca75:online}
un sistema formale definito da Alonzo Church nel 1936, sviluppato
per definire formalmente le funzioni e il loro calcolo.

In un paradigma basato su stati come la programmazione ad oggetti
spesso si hanno variabili condivise mutabili, ovvero, più parti del codice
possono riferirsi alla stessa variabile, e questo complica la programmazione
multithreading dovendosi preoccupare di meccanismi come il blocco
sincronizzato o il locking per evitare le race condition tra più parti
del codice, e non è immediato scrivere del codice concorrenziale sicuro
e spesso si riscontrano comportamenti indeterminati.
In un paradigma funzionale si predilogono le variabili immutabili che
aggirano questo problema riducendo il rischio di scrivere codice
concorrenziale non sicuro.

Cambiare paradigma non è immediato, un paradigma si può dire che
definisce il modo di pensare al problema, nella programmazione ad
oggetti per esempio si definiscono le cosidette classi,
pensando al problema come oggetti che possono comportarsi in un
determinato modo attraverso le funzioni definite su di esso.
Perciò si pensa ad un oggetto che ha un comportamento e che cambia
il suo stato nel tempo, un modo di sviluppare intuitivo ma non sempre
ottimale per la risoluzione di problemi. Nella programmazione funzionale
si cambia prospettiva, ovvero si ha un input, si passa l'input 
alla funzione e si ottiene la trasformazione dell'input ottenendo
l'output.

In poche parole un linguaggio funzionale assume che scrivere un
software complesso sia più facile nel momento in cui il codice ha
queste proprietà:
\begin{itemize}
    \item I dati sono immutabili
    \item Le funzioni sono pure, ovvero, il risultato di una funzione
    dipende soltanto dai suoi parametri in input.
    \item Le funzioni non generano effetti oltre il suo valore restituito.
\end{itemize}

Con queste proprietà si ha più controllo del flusso del programma,
anche se non sempre possono essere soddisfatte.

%------------------------------------------------------------------

\subsection{Struttura di un progetto Elixir}

Elixir è un linguaggio moderno, e come ogni linguaggio moderno che
si rispetti fornisce un tool per la creazione e configurazione di
progetti, questo tool si chiama \textbf{Mix}.

%------------------------------------------------------------------

\subsubsection{Il tool Mix}
È possibile creare un progetto con il comando:

\begin{lstlisting}[language=none]
mix new <nome-progetto>
\end{lstlisting}

Verrà creata una struttura per il progetto come nell'esempio \ref{lst:struttura_progetto},
il codice sarà organizzato nella cartella \textbf{lib}, viene creata una
cartella \textbf{test}, e il file per la configurazione del progetto \textbf{mix.exs}.
Da notare che anche la configurazione del progetto avviene tramite funzioni.

\begin{lstlisting}[language=none,captionpos=b,caption={Struttura progetto},label={lst:struttura_progetto}]
.
>build
>deps
  -->...
>lib
  -->example.ex
mix.exs
README.md
>test
  -->example.exs
\end{lstlisting}

Come sappiamo Elixir fornisce anche un ambiente interattivo (\textbf{iex}) per
testare il nostro codice, ed è consentito avviare questo ambiente nel
dominio della nostra applicazione con il comando:

\begin{lstlisting}[language=none]
iex -S mix 
\end{lstlisting}
Si può compilare il progetto con:
\begin{lstlisting}[language=none]
mix compile
\end{lstlisting}

Con Mix possiamo includere e scaricare facilmente anche librerie esterne attraverso
il package manager.\cite{HexDocs99:online}

\subsubsection{Moduli}
Elixir organizza il codice in Moduli, permettendo di definire
le funzioni dentro dei namespace, così da separare
le responsabilità delle funzioni.

Ci sono varie cose che si possono definire dentro un modulo,
si possono definire delle \textbf{struct}
ma cosa più importante si possono definire i cosiddetti 
\textbf{Behaviour}, un modo per definire un interfaccia Api,
Elixir fornisce delle astrazioni proprio attraverso questi.
%TODO menzionare GenServer e Supervisor
Ciò che si vuole evidenziare ora è che il progetto è definito
in moduli, il modo che Elixir fornisce per organizzare il
codice.

% ----------------------------------------------------------------------------

\subsection{Basi dichiarative}

Come già accennato, Elixir adotta un approccio
dichiarativo nella definizione delle funzioni.
Questo si contrappone all'approccio imperativo,
che si concentra su "come posso risolvere questo problema?",
mentre quello dichiarativo si pone la domanda 
"come posso definire un problema?".

Nell'esempio \ref{lst:somma_n_cpp} è presentato un approccio
imperativo al problema "somma dei primi n elementi", mentre
nell'esempio \ref{lst:somma_n_elixir} è presentato l' approccio dichiarativo
con Elixir.


\begin{lstlisting}[language=cpp, caption={Somma N elementi},captionpos=b,
	label={lst:somma_n_cpp}]
int sum_first_n(n){
  int sum=0;
    for(int i=1;i++;i<=n){
      sum+=i;
    }
    return sum;
}
\end{lstlisting}

\begin{lstlisting}[language=elixir, caption={Somma N elementi},captionpos=b,
	label={lst:somma_n_elixir}]
defmodule Sum do
  def sum_recursive(0), do: 0
  def sum_recursive(n), do: n + sum_recursive(n - 1)
end
\end{lstlisting}

In particolare questo approccio è rafforzato tramite il meccanismo
del \textbf{Pattern Matching}, infatti in Elixir l'operatore $=$,
non è un operatore di assegnazione, ma è comparabile
all'equivalente algebrico.
Quest'operatore ci permette di scrivere dell'equazioni
che condizionano il flusso del codice.
Questo è molto utile per l'approccio
dichiarativo, infatti possiamo definire più corpi della stessa funzione
ed Elixir capisce quale funzione invocare in base
al valore dei suoi parametri attraverso questo meccanismo,
è già stato utilizzato implicitamente nell'esempio \ref{lst:somma_n_elixir},
dove viene invocata la funzione "sum\_recursive(0)" quando
il valore dell'argomento vale 0, altrimenti chiamerà quello
con l'argomento "n" assegnando il valore dell'argomento alla variabile n.

Elixir cerca di assegnare il valore a destra dell'equazione
al valore di sinistra cercando di risolvere l'equazione tentando
di fare un assegnazione dove possibile.
Questo ci permette di usare quest'operatore per poter
scomporre un dato, un esempio può essere usato usando l'ambiente
interattivo \textit{iex}

\begin{lstlisting}[language=elixir, caption={Pattern Matching},captionpos=b,
	label={lst:pattern_matching}]
# Lists
iex> list = [1, 2, 3]
[1, 2, 3]
iex> [1, 2, 3] = list
[1, 2, 3]
iex> [] = list
** (MatchError) no match of right hand side value: [1, 2, 3]
iex> [1 | tail] = list
[1,2,3]
iex> tail
[2,3]
\end{lstlisting}

%--------------------------------------------------------------------------

\subsection{Transizione al funzionale - Immutabilità dei dati}

Cambiare paradigma può essere difficoltoso, bisogna
cambiare prospettiva, ma con le giuste intuizioni
può risultare semplice. Per fare una transizione al funzionale
bisogna capire soprattuto qual'è la differenza rispetto ad un
linguaggio basato sugli stati.

Elixir non ha oggetti, il linguaggio ha un forte focus sull'immutabilità.
In Elixir trasformiamo dati piuttosto che mutarli, infatti da questo
punto di vista probabilmente è più naturale il paradigma funzionale
piuttosto che un procedurale, ad esempio in un linguaggio
procedurale possiamo scrivere:

\begin{lstlisting}[language=cpp]
my_array = [1,2,3]
do_something_with_array(my_array)
print(my_array)
\end{lstlisting}

Ci potremmo aspettare che la stampa dell'array sia [1,2,3] quando in realtà
l'output dipende da cosa fa la funzione. Mentre in Elixir questa cosa
non è implicita, ma l'assegnazione deve essere sempre esplicita come
segue:

\begin{lstlisting}[language=elixir]
my_list = [1,2,3]
my_list = do_something_with_array(my_list)
IO.inspect(my_list)
\end{lstlisting}

Se my\_list non viene riassegnata, my\_list non cambia.
Il problema può arrivare nel momento in cui ci serve uno stato
da condividere, una semplice variabile globale non esiste
in Elixir, e per mantenere uno stato dobbiamo
affidarci ad un altro processo che lo mantiene per noi,
ed Elixir almeno per questo ci da delle astrazioni
a cui fare affidamento come il Modulo \textbf{Agent} o il
\textbf{GenServer} che verranno trattati in seguito.

%--------------------------------------------------------------------------
\subsection{Conseguenze sulle prestazioni dell'immutabilità}

L'immutabilità ci consente di avere un controllo maggiore
sul flusso del codice, questo però può portare ad un'inefficienza
dovendo sempre copiare dati, inoltre si può pensare che si creano
molti dati non più utilizzati da liberare in memoria con il Garbage collector.
Sfruttando questa proprietà però si possono limitare i danni.

Elixir sa che i dati sono immutabili, e può riusarli in parte o per
intero quando si creano nuove strutture.
Il seguente esempio mostra come Elixir ottimizza la creazione di
nuovi dati:

\begin{lstlisting}[language=elixir]
iex> lista1 = [ 3, 2, 1 ]
[3, 2, 1]
iex> lista2 = [ 4 | lista1 ]
[4, 3, 2, 1]
\end{lstlisting}

La lista2 infatti è creata usando la lista1 sapendo che non
cambierà mai durante il flusso di esecuzione, quindi mette
semplicimente in coda la lista1.

\subsubsection{Garbage Collection}
Un altro aspetto che può creare dubbi sul copiare dati, è che
si lasciano spesso vecchi dati non più utilizzati da dover
pulire con il Garbage Collector.

Scrivendo codice Elixir però ci rendiamo subito conto
che ci porta a sviluppare codice con migliaia di processi leggeri,
ed ogni processo ha il proprio heap. I dati nell'applicazione quindi
sono suddivisi nei processi, e ogni heap è molto piccolo rendendo
il garbage collector molto veloce. Se un processo termina prima che
il suo heap diventi pieno, tutti i suoi dati vengono eliminati, senza necessità
di liberare la memoria rendendo il garbage collector
efficiente. \cite{programming_elixir_immutability}

