\section{Introduzione}



In questo capitolo, esamineremo le caratteristiche distintive
di Elixir, un linguaggio di programmazione funzionale e
concorrente che sfrutta appieno la potenza della piattaforma
OTP (Open Telecom Platform), non si vuole coprire ogni dettaglio
del linguaggio, ma mettere in evidenza ciò che fa di Elixir
un buon linguaggio.

Elixir, scritto in Erlang ed eseguito sulla macchina virtuale
Erlang (BEAM), eredita gli obiettivi di Erlang, ma apporta
miglioramenti significativi per rendere il linguaggio più
appetibile e moderno.

Erlang, nato nel 1986, è stato progettato per semplificare
lo sviluppo di software concorrente e robusto.
Elixir si basa su queste fondamenta solide, offrendo un'API
più pulita e astrazioni avanzate che consentono ai programmatori
di ragionare a un livello più elevato, facilitando la scrittura
di codice concorrente in modo intuitivo.

Una delle massime principali di Erlang e, di conseguenza,
di Elixir, è "Let it crash" (Lascia che si schianti),
che riflette l'approccio alla gestione degli errori nei sistemi
concorrenti, incoraggiando la gestione degli errori tramite
il rilancio e la supervisione anziché il blocco del processo.

Per capire come lavorare con questo linguaggio, bisogna affrontare
un po' di questioni e farsi un po' di domande.
Bisogna capire come la macchina virtuale Beam affronta la concorrenza,
Elixir in particolare è un linguaggio orientato alla concorrenza e 
le astrazioni che fornisce sono proprio per far sì che
si programmi in modo concorrenziale portando ad avere un
codice responsivo e gestendo bene i processi attraverso 
il meccanismo di Supervision, il software diventa anche robusto.
Un altro punto da affrontare è l'immutabilità dei dati, è un concetto
chiave in Elixir ed Erlang, è proprio questa caratteristica
che ci semplifica la programmazione concorrenziale.
