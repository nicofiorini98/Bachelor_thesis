\newpage

\section{Sistemi distribuiti}


Nell'era dell'informatica moderna, la necessità di
sviluppare applicazioni scalabili, affidabili e resilienti
è diventata sempre più critica.
In risposta a questa esigenza, i sistemi distribuiti sono
emersi come una soluzione efficace per gestire carichi
di lavoro complessi su più nodi di calcolo,
consentendo alle applicazioni di crescere e adattarsi
dinamicamente alle esigenze in continua evoluzione degli utenti.

Elixir offre un ambiente potente e flessibile
per lo sviluppo di sistemi distribuiti.
Grazie al suo modello di programmazione ad attori
e al supporto integrato per la comunicazione distribuita,
Elixir è ideale per la creazione di applicazioni distribuite
altamente scalabili, affidabili e resilienti.

Utilizzando il modello di programmazione ad attori,
le applicazioni possono essere decomposte in processi leggeri(attori)
che comunicano tra loro attraverso lo scambio di messaggi asincroni.
Questo modello semplifica la gestione della concorrenza
e delle comunicazioni tra nodi, consentendo alle applicazioni di
gestire carichi di lavoro elevati in modo efficiente e affidabile.

La robustezza è sicuramente un obiettivo primario nei progetti
distribuiti, e Elixir fornisce strumenti e astrazioni che semplificano
la gestione degli imprevisti e dei guasti.
La capacità di inviare messaggi in remoto tra i nodi,
utilizzando gli stessi meccanismi utilizzati per la comunicazione
locale, è un esempio di come Elixir semplifica lo sviluppo di
sistemi distribuiti, garantendo al contempo affidabilità e resilienza.

È importante sottolineare che molti dei concetti e delle pratiche
utilizzate nello sviluppo di sistemi distribuiti in Elixir
sono ereditati dall'Erlang Runtime, il che garantisce una base
solida e comprovata per la costruzione di applicazioni distribuite.

Uno dei principali vantaggi dei sistemi distribuiti
in Elixir è la capacità di scalare orizzontalmente aggiungendo
nuovi nodi al cluster. 
Ciò consente alle applicazioni di crescere in modo flessibile
con l'aumento del traffico e di garantire un servizio continuo
agli utenti finali anche in caso di errori o guasti dei nodi.

\subsection{Un esempio di comunicazione tra nodi}

Un nodo in ogni sistema Erlang può essere identificato
con un nome. Ad esempio si può avviare l'abiente iex
assegnando un nome all'istanza.

\begin{lstlisting}[language=elixir]
iex --sname nico@localhost

iex(nico@localhost)1> 
	
\end{lstlisting}

Si può avviare un altra istanza:

\begin{lstlisting}[language=elixir]
iex --sname kate@localhost
	
iex(nico@localhost)1> 
\end{lstlisting}

Questi possono essere identificati come due nodi
e comunicare tra loro tramite Node.spawn\_link/2,
prende due argomenti, il nome del nodo al quale connettersi,
e la funzione da eseguire dal processo remoto.

Per la comunicazione tra i due nodi si usano le
stesse primitive usate per la concorrenza, un nodo
può essere visto come un processo.

\subsubsection{Comunicazione su nodi in remoto}

Per mandare messaggi e riceverli tra due nodi
su macchine differenti, le istanze devono
essere avviate con un cookie come segue:

\begin{lstlisting}[language=elixir]
iex --sname nico@localhost --cookie secret_token
\end{lstlisting}

Solo i nodi che condividono lo stesso cookie hanno
il permesso di comunicare.

Un altro modo per comunicare è tramite Node.connect/1,
i nodi si connettono tra loro, ed Elixir mantiene un
connessione TCP aperta tra i due nodi.
Se altri nodi si connettono, si mantengono connessioni
l'un l'altro, si forma così una rete denominata `fully meshed network`.


\subsection{Considerazioni e sfide da considerare}

La gestione delle connessioni tra i nodi distribuiti può
essere complessa e non viene approfondita in questo luogo.
La programmazione distribuita è un campo complesso ed Elixir
prova a semplificarla usando le stesse primitive usate per la
concorrenza, ma possono sorgere dei problemi.

La gestione delle connessioni tra i nodi distribuiti può
essere complessa.
È necessario garantire che i nodi possano comunicare
tra loro in modo affidabile e che le connessioni
rimangano stabili anche in presenza di interruzioni
di rete o guasti hardware.

La comunicazione tra i nodi può essere
vulnerabile agli attacchi di rete, quindi è importante
implementare misure di sicurezza per proteggere i dati
e prevenire accessi non autorizzati.
Ciò include l'uso di connessioni sicure, autenticazione
e autorizzazione dei nodi, e crittografia dei dati trasmessi.

La tolleranza agli errori è fondamentale nei sistemi distribuiti.
È necessario gestire i guasti dei nodi in modo che
l'applicazione possa continuare a funzionare in modo affidabile
anche in presenza di errori o interruzioni di servizio.

La scalabilità è un altro aspetto importante nella
programmazione distribuita.
È necessario progettare il sistema in modo che possa
gestire carichi di lavoro elevati e crescere in modo
flessibile con l'aumento del traffico senza compromettere
le prestazioni o l'affidabilità.


