\section{Struttura di un progetto Elixir}

Elixir è un linguaggio moderno, e come ogni linguaggio moderno che
si rispetti fornisce un tool per la creazione e configurazione di
progetti, questo tool si chiama \textbf{Mix}.

\subsection{Il tool Mix}
È possibile creare un progetto con il comando:

\begin{lstlisting}[language=none]
mix new <nome-progetto>
\end{lstlisting}

Verrà creata una struttura per il progetto, e nel progetto saranno
presenti una cartella \textbf{lib} dove andrà il codice,
una cartella \textbf{test} per creare degli Unit Test, ma soprattutto
viene creato un file \textbf{mix.exs} per la configurazione del progetto.

Come sappiamo Elixir fornisce anche un ambiente interattivo (\textbf{iex}) per
testare il nostro codice, ed è consentito avviare questo ambiente nel
dominio della nostra applicazione con il comando:

\begin{lstlisting}[language=none]
iex -S mix 
\end{lstlisting}
Si può compilare il progetto con:
\begin{lstlisting}[language=none]
mix compile
\end{lstlisting}

Con Mix possiamo includere e scaricare facilmente anche librerie esterne attraverso
il package manager.\cite{HexDocs99:online}

\subsection{Moduli}
Elixir organizza il codice in Moduli, che permettono di definire
le funzioni dentro dei namespace,permettendoci così di separare
le responsabilità raggruppando le funzioni.

Ci sono varie cose che si possono definire dentro un modulo,
si possono definire delle \textbf{struct}
ma cosa più importante si possono definire i cosiddetti 
\textbf{Behaviour}, un modo per definire un interfaccia Api,
Elixir fornisce delle astrazioni proprio attraverso questi.
%TODO menzionare GenServer e Supervisor
Ciò che si vuole evidenziare ora è che il progetto è definito
in moduli, ed è il modo che Elixir fornisce per organizzare il
codice.




