\section{Struttura di un progetto Elixir}

Elixir è un linguaggio moderno, e come ogni linguaggio moderno che
si rispetti fornisce un tool per la creazione e configurazione di
progetti, questo tool si chiama \textbf{Mix}.
È possibile creare un progetto con il comando:
% \begin{lstlisting}[language=bash]
% 	echo "Hello, world!"
% \end{lstlisting}

% \begin{minted}{bash}
% 	#!/bin/bash
% 	echo the $# parameter did not destroy pygments syntax highlighting
% \end{minted}

\begin{lstlisting}[language=bash]
mix new <nome-progetto>
\end{lstlisting}

\begin{lstlisting}[language=elixir]
	mix new <nome-progetto>
\end{lstlisting}

\begin{lstlisting}[language=elixir, caption={Super fancy code},captionpos=b]
# A comment
defmodule Foo do
    def bar(x) do
        string = "hello"
        fc = fn x,y -> x + y end
        atom = fc.({:imanatom})
        atom = fc.({:because})
    end
end
\end{lstlisting}
