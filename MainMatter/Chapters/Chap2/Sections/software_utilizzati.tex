\section{Software utilizzati}

I studi sulle performance riportati nel capitolo 3
sono stati effettuati
utilizzando il sistema operativo Linux per scelta personale e
per la varietà di software disponibile su Linux.
Viene riportata l'installazione effettuata su Linux per i vari test e le
relative configurazioni, in modo da poter testare gli esempi
riportati in questo capitolo, l'installazione su Windows è immediata,
infatti è disponibile sia l'installer per Erlang al seguente link: \url{https://www.erlang.org/downloads.html}
e l'installer anche per elixir nel sito ufficiale \url{https://elixir-lang.org/install.html},
mentre su linux necessita qualche configurazione in più e viene riportata
l'installazione tramite asdf.

\subsection{Installazione software utilizzati} 

Per avere l'ultima versione di Erlang ed Elixir su Linux bisogna affidarsi
al Tool version manager \textbf{asdf} in quanto le versioni disponibili nei
package manager della distro di riferimento non sono aggiornate all'ultima
versione.

\subsubsection{Installazione asdf}

Asdf è un tool version manager open source che consente di installare,
gestire e utilizzare diverse versioni di software senza dover
installare o gestire separatamente ciascuna versione. È soprattutto
utile quando si valora su progetti che richiedono versioni specifiche
del linguaggio. Nel nostro caso ci permette di scegliere l'ultima
versione disponibile su qualunque distro linux indipendentemente
dalla distro scelta.

Per l'installazione di asdf basta seguire le istruzioni della
documentazione ufficiale del tool al seguente URL: \url{https://asdf-vm.com/guide/getting-started.html}

Le dipendenze necessarie da installare sono \textbf{curl} e \textbf{git},
una volta installate le dipendenze richieste basta clonare la repository.
In una distro debian i comandi risultano i seguenti:

% \begin{lstlisting}[language=none,captionpos=b,caption={Struttura progetto},label={lst:struttura_progetto}]
\begin{lstlisting}[language=none]
> sudo apt install curl git
> cd $HOME
> git clone https://github.com/asdf-vm/asdf.git ~/.asdf --branch v0.14.0
\end{lstlisting}

Inserire le seguenti righe nel file di configurazione \textit{.bashrc}:

\begin{lstlisting}[language=none]
. "$HOME/.asdf/asdf.sh"
. "$HOME/.asdf/completions/asdf.bash"
\end{lstlisting}

Assicurarsi che l'installazione che sia avvenuta con successo:

\begin{lstlisting}[language=none]
> asfd --version

v0.14.0
\end{lstlisting}

%---------------------------------------------------------------
\subsubsection{Installazione Erlang}

Una volta installato asdf, installare Erlang con la versione desiderata
seguendo le istruzioni della seguente repository: \url{https://github.com/asdf-vm/asdf-erlang}
Per una distro ubuntu si richiedono le seguenti dipendenze da installare con
il package manager apt:
\begin{lstlisting}[language=none]
sudo apt-get -y install build-essential autoconf m4 libncurses5-dev
libglu1-mesa-dev libpng-dev libssh-dev unixodbc-dev xsltproc
fop libxml2-utils libncurses-dev openjdk-11-jdk
\end{lstlisting}

Per evitare warning nell'intellisense di VScode, prima di procedere con i
comandi di asdf, bisgona configurare
le seguenti variabili d'ambiente per compilare la documentazione di Erlang
necessaria al corretto funzionamento dell'intellisense di VSCode:

\begin{lstlisting}[language=none]
> export KERL_BUILD_DOCS=yes
> export KERL_INSTALL_HTMLDOCS=yes
> export KERL_INSTALL_MANPAGES=yes
\end{lstlisting}

Facendo attenzione ad usare la stessa sessione del terminale in cui sono
stati eseguiti i precedenti comandi eseguire:

\begin{lstlisting}[language=none]
> asdf plugin add erlang https://github.com/asdf-vm/asdf-erlang.git
> asdf install erlang 26.2
> asdf global erlang 26.2 
\end{lstlisting}

Si può anche scegliere una versione differente di Erlang, fare attenzione
alla versione dell'OTP che viene installata in quanto sarà
necessaria per l'installazione di Elixir.

Se tutto è andato a buon fine si può testare l'avvenuta installazione
tramite il comando \textit{erl} che avvia l'ambiente interattivo di erlang.

%------------------------------------------------------
\subsubsection{Installazione Elixir}

Per Elixir il procedimento con asdf è simile, bisogna solo fare
attenzione alla versione OTP di Erlang installata, in quanto
Elixir viene compilato con diverse versioni dell'OTP,
fare riferimento al seguente link per il procedimento di installazione:
\url{https://github.com/asdf-vm/asdf-elixir}

Installare le dipendenze richieste:
\begin{lstlisting}[language=none]
> sudo apt-get install unzip
\end{lstlisting}

Eseguire i seguenti comandi per procedere con l'installazione:
\begin{lstlisting}[language=none]
> asdf plugin-add elixir https://github.com/asdf-vm/asdf-elixir.git
> asdf install elixir 1.16.0-otp-26
> asdf global elixir 26.2 
\end{lstlisting}

Se tutto è andato a buon fine si può testare elixir avviando l'ambiente
interattivo di elixir \textbf{iex}.

%---------------------------------------------------------------------------

\subsubsection{Configurazione VsCode}

VsCode supporta numerosi linguaggi di programmazione, è gratuito e
open source, è un editor leggero che consente un esperienza di sviluppo
piacevole e produttiva con una vasta gamma di estensioni installabili.
Per quanto riguarda Elixir, è possibile installare l'estensione
\textbf{ElixirLS} disponibile al seguente link:
\url{https://marketplace.visualstudio.com/items?itemName=JakeBecker.elixir-ls}
 