\chapter*{Conclusioni e sviluppi futuri}
\addcontentsline{toc}{chapter}{Conclusioni e sviluppi futuri}

\setlength{\parindent}{0pt}

Elixir si è dimostrato un linguaggio potente che
migliora molti aspetti dello sviluppo software,
soprattutto quello in ambito concorrenzile, sviluppare
è stato piacevole dimostrando di essere un linguaggio
moderno degno di nota.
\vspace{1.5em}

Ha dimostrato di avere uno scheduling molto
veloce che non degrada significamente le performance.
Se si necessita di computazioni dove Elixir non
ottiene prestazioni ottimali 
si può integrare con 
soluzioni più adatte come il C/C++, Rust o Python, tramite
inter process communication o tramite funzioni native NIF.
Ha dimostrato di avere una bassa latenza ottimale per
sistemi real time, ed è quello il suo principale utilizzo
adottato finora, ma ha dimostrato di essere una scelta promettente
anche nell'ambito dell'IoT, dove affidabilità reattività 
e concorrenza sono dei fattori chiave.
\vspace{1.5em}

Futuri studi possono vedere l'integrazione di Elixir
in software IoT, tramite il framework Nerves o con
soluzioni ad Hoc per il caso specifico.