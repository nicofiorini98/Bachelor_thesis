\section{Introduzione}
In questo capitolo si vuole capire quali potenzialità
la piattaforma di Erlang/Elixir porta, in modo da poter
capire quali siano le applicazioni possibili con questa tecnologia.

Vogliamo capire i limiti di questa tecnologia, attraverso una serie
di test empirici e provare a fare qualche confronto con altri
linguaggi.

% ---------------------Hardware utilizzato-------------------------- 
\subsection{Hardware utilizzato}

I test vengono eseguiti sul sistema operativo linux, viene riportato in
figura \ref{fig:neofetch} l'output del tool \textbf{neofetch} che riporta
le caratteristiche del computer utilizzato per l'esecuzione dei test,
in particolare notiamo che la CPU di riferimento dei test
ha 8 core logici e 4 fisici, questa caratteristica è importante
in quanto ci aspettiamo che i test concorrenti diano il meglio di
sè con più di 4 processi attivi.

\begin{figure}[!htp]
    \centering
    \includegraphics[keepaspectratio=true,scale=0.3]{images/neofetch.png}
	\caption{Output neofetch}
  	\label{fig:neofetch}
\end{figure}

% ---------------------Test Interoperabilità-------------------------- 


