\section{Test concorrenziale}\label{sec:test_concorrenziale}

Questo test segue dal lavoro effettuato dalla mia collega
Luisa Fausta D'Epiro, è una rivisitazione dei risultati
ottenuti con qualche modifica e reinterpretazione dei risultati.\cite{tesiLuisa}, 

In questo Test empirico si vuole vedere come Elixir
si comporta all'aumentare dei processi, per capire
se lo scheduling rappresenta un collo di bottiglia
all'aumentare dei processi utilizzati.

In particolare si sono scelti una lista di campioni
di processi (\textit{processes}) da utilizzare per ogni test,
per ogni valore della lista \textit{processes}
si eseguono una serie prodotti che vanno da 500 prodotti
a 100000 prodotti con uno step di 500.

Si calcola quindi il tempo impiegato per ogni campione
e si stampa il campione sul un file csv da poter analizzare
ed interpretare successivamente con Matlab.

I campioni di processi da creare per ogni test
sono riportati nella lista:
\begin{lstlisting}[language=none]
processes = [1,2,3,4,5,6,7,8,16,32,64,128,256]
\end{lstlisting}

\subsection{Implementazione del Test}

Il codice Elixir che esegue il test è
implementato nel modulo Elixir ConcurrentTask e
riportato nel Listing \ref{lst:concurrent_task}.


\begin{lstlisting}[language=elixir, caption={Test concorrenziali},captionpos=b,
	label={lst:concurrent_task}]
defmodule ConcurrentTask do
  require Logger
  import MyFile


  def compute_products(productsnumber) do
    for _i <- 1..productsnumber do
  	1 * 1000
    end
  end


  def run do
    processes = [1, 2, 3, 4, 5, 6, 7, 8, 16, 32, 64, 128, 256]
    productsnumber = 100_000
    step = 500
  
    for proc <- processes do
  	Logger.info("processes #{proc} and #{System.schedulers} scheduler")
  	for comp <- 500..productsnumber//step do
  	  {:ok, _time} = parallel_operations(comp, proc)
  	end
    end
  end

	def parallel_operations(productsnumber, processnumber) do
  
    	#divide the products number to assign to each process
		temp = trunc(productsnumber / processnumber)
  		# compute the rest to compute to restTask
  		rest = rem(productsnumber , processnumber)
  
  		{time, _result} =
  	  	:timer.tc(
	  		fn ->
  		  	tasks =
  				for _i <- 1..processnumber do
  			  	Task.async(fn -> compute_products(temp) end)
  				end
  		  	restTask = Task.async(fn -> compute_products(rest) end)
  
  		  	# Aspetta di finire ogni task per ottenere i risultati
  		  	for task <- tasks do
  				Task.await(task, :infinity)
  		  	end
  		  	Task.await(restTask, :infinity)
  			end,
  			[],
  			:microsecond
  	  	)
  	  	writeData2File(time,processnumber,productsnumber)
  	  	{:ok,time}
	end

	def writeData2File(time,processnumber,productsnumber) do
  	available_scheduler = 
		:erlang.system_info(:logical_processors_available)

	scheduler = System.schedulers()

  		data = [
		"#{scheduler},",
		"#{available_scheduler},",
		"#{time},",
		"#{processnumber},",
		"#{productsnumber}\n"
  	]

	# scrittura risultato su file
	write(data)
  	{:ok, time}
	end
end
  
\end{lstlisting}

La funzione write che si occupa della scrittura
dei dati su file si trova nel modulo MyFile riportato nel
Listing \ref{lst:MyFile}.

\newpage

\begin{lstlisting}[language=elixir, caption={Modulo MyFile},
	captionpos=b,label={lst:MyFile}]
defmodule MyFile do
	def write(data,file_path \\ "./File/test.csv") do
		{:ok, file} = File.open(file_path, [:write,:append])
		IO.write(file, data)
    	File.close(file)
	end 
end
\end{lstlisting}

Il modulo Task utilizzato fornisce un modo per eseguire
una funzione in background e recuperarne il valore
restituito in un secondo momento.
Il modulo Task fornito è
implementato utilizzando la primitiva spawn\_link discussa
in precedenza.

%++++++++++++++++++++++++++++++++++++++++++++++++++++++++++

\subsection{Esecuzione Test}

Si è creato uno script per l'ambiente interattivo iex per l'avvio
della funzione voluta riportato nel Listing \ref{lst:script_iex},
viene compilato il modulo concurrentTask ed eseguita la funzione
run().

\begin{lstlisting}[language=none,captionpos=b,
	caption={Script iex per l'avvio dei test},
	label={lst:script_iex}]
:code.purge(ConcurrentTask)
:code.delete(ConcurrentTask)
c("lib/concurrent_task.ex")
ConcurrentTask.run
System.halt
\end{lstlisting}
	

Il test fornito è stato eseguito più volte aumentando
gli scheduler allocati all'avvio dell'istanza della VM.
Si è scritto un semplice script in bash per eseguire
i vari test all'aumentare degli scheduler allocati, lo script
è riportato nel Listing \ref{lst:script_bash}.
La macchina virtuale di default viene allocata con 8
scheduler, si avvia con il numero di scheduler voluti
impostando il flag --erl "+S1", per l'avvio con più
di 8 scheduler si deve forzare la macchina virtuale
a farlo con il flag "+sbt db".

\begin{lstlisting}[language=none,captionpos=b,
	caption={Script bash per l'avvio dei test},
	label={lst:script_bash}]

#!/bin/bash

iex --erl "+S 1" --dot-iex "runTest.iex" -S mix
iex --erl "+S 2" --dot-iex "runTest.iex" -S mix
iex --erl "+S 3" --dot-iex "runTest.iex" -S mix
iex --erl "+S 4" --dot-iex "runTest.iex" -S mix
iex --erl "+S 5" --dot-iex "runTest.iex" -S mix
iex --erl "+S 6" --dot-iex "runTest.iex" -S mix
iex --erl "+S 7" --dot-iex "runTest.iex" -S mix
iex --erl "+S 8" --dot-iex "runTest.iex" -S mix
iex --erl "+S 9 +sbt db" --dot-iex "runTest.iex" -S mix
iex --erl "+S 10 +sbt db" --dot-iex "runTest.iex" -S mix
iex --erl "+S 11 +sbt db" --dot-iex "runTest.iex" -S mix
iex --erl "+S 12 +sbt db" --dot-iex "runTest.iex" -S mix
iex --erl "+S 13 +sbt db" --dot-iex "runTest.iex" -S mix
iex --erl "+S 14 +sbt db" --dot-iex "runTest.iex" -S mix
iex --erl "+S 15 +sbt db" --dot-iex "runTest.iex" -S mix
iex --erl "+S 16 +sbt db" --dot-iex "runTest.iex" -S mix
	
\end{lstlisting}

Per avviare lo script eseguire i comandi:
\begin{lstlisting}[language=none]
# solo la prima volta alla creazione del file
chmod +x <directory-path>/runtest.sh 

.<directory-path>/runtest.sh
\end{lstlisting}

Una volta avviati i test verrà creato un file .csv da
analizzare con Matlab, i valori presenti nel file
assumono la forma:

\begin{lstlisting}[language=none,captionpos=b,
	caption={File csv} \label{lst:formatocsv}]
N_Scheduler,N_Available_Scheduler,Time,N_Processes,N_Products
1,8,39,1,500
1,8,24,1,1000
1,8,19,1,1500
1,8,20,1,2000
1,8,22,1,2500
.....
\end{lstlisting}


%++++++++++++++++++++++++++++++++++++++++++++++++++
\subsection{Analisi Matlab}

In Matlab viene analizzato il file .csv risultante dal
test, stampando un grafico per ogni numero di scheduler utilizzato.
Lo script Matlab è riportato nel Listing \ref{lst:analisi_matlab}.

\begin{lstlisting}[language=none,captionpos=b,
	caption={Codice Matlab per la stampa dei grafici},label={lst:analisi_matlab}]
opts = detectImportOptions('<replace-with-filecsv-path>');
opts.DataLine = 2;
data = readtable('<replace-with-filecsv-path>', opts);

colors = [
    255 0   0   % Red
    0   255 0   % Green
    0   0   255 % Blue
    255 255 0   % Yellow
    0   255 255 % Cyan
    255 0   255 % Magenta
    128 128 128 % Gray
    255 128 0   % Orange
    128 0   128 % Purple
    0   128 128 % Teal
    128 128 0   % Olive
    255 165 0   % Orange (Web Color)
    0   255 127 % Spring Green
    ];
colors = colors / 255;
processes = [1,2,3,4,5,6,7,8,16,32,64,128,256];

for n = 1:16

    figure;

    for i = 1:length(processes)

        num_processes = processes(i);

        % Filtra i dati per N_Processes = 1 e N_Products = 1
        filteredData = ... 
		data(data.N_Processes == num_processes ... 
		     & data.N_Scheduler==n,:);

        filteredTime = ... 
		(filteredData.Time ./filteredData.N_Products);
        plot(filteredData.N_Products, filteredTime,... 
		     'Color', colors(i, :));

        hold on

    end
    xlabel('N_Products');
    ylabel('Time/Products');

    title('Grafico per n. Scheduler: ',n);


    legend('1 process','2 processes','3 processes','4 processes', ...
        '5 processes','6 processes','7 processes','8 processes', ...
        '16 processes','32 processes','64 processes',... 
		'128 processes','256 processes');
end
\end{lstlisting}

Lo script Matlab stampa 16 grafici, ne vengono riportati
tre, per l'esecuzione con 1 scheduler in figura \ref{fig:1_scheduler},
,con 4 scheduler in figura \ref{fig:4_scheduler} per 8 scheduler in figura \ref{fig:8_scheduler}
e per 16 in figura \ref{fig:16_scheduler}.

\begin{figure}[!htp]
    \centering
    \includegraphics[keepaspectratio=true,scale=0.33]{images/matlab/1_scheduler.png}
	\caption{Grafico con 1 scheduler}
  	\label{fig:1_scheduler}
\end{figure}

\begin{figure}[!htp]
    \centering
    \includegraphics[keepaspectratio=true,scale=0.335]{images/matlab/4_scheduler.png}
	\caption{Grafico con 4 scheduler}
  	\label{fig:4_scheduler}
\end{figure}

\begin{figure}[!htp]
    \centering
    \includegraphics[keepaspectratio=true,scale=0.335]{images/matlab/8_scheduler.png}
	\caption{Grafico con 8 scheduler}
  	\label{fig:8_scheduler}
\end{figure}

\begin{figure}[!htp]
    \centering
    \includegraphics[keepaspectratio=true,scale=0.335]{images/matlab/16_scheduler.png}
	\caption{Grafico con 16 scheduler}
  	\label{fig:16_scheduler}
\end{figure}

I test risultano avere degli andamenti simili, possiamo notare che
con l'effettuazione di pochi prodotti, aumentare il numero dei processi
porta ad un degradamento delle performance, in quanto la VM si occupa più
di schedulare i processi che a fare prodotti, uno zoom di questo
andamento è riportato in figura \ref{fig:zoom_8_inizio}.
Per pochi prodotti come previsto lo scheduling fa perdere efficienza
alle operazioni svolte.
Però si nota che l'andamento con pochi processi lo scheduling non porta
ad una mancanza di efficienza nonostante le operazioni da svolgere prendono
pochissimo tempo.
\begin{figure}[!htp]
    \centering
    \includegraphics[keepaspectratio=true,scale=0.335]{images/matlab/zoom_8_inizio.png}
	\caption{Zoom con 8 scheduler, range 500-10000 prodotti}
  	\label{fig:zoom_8_inizio}
\end{figure}

Facendo uno zoom per un numero di prodotti più importante, come in
figura \ref{fig:zoom_8_fine}, si nota che con un processo le operazioni
da svolgere risultano più lente, e già con più processi la parallelizzazione
inizia ad avere più senso.

\begin{figure}[!htp]
    \centering
    \includegraphics[keepaspectratio=true,scale=0.33]{images/matlab/zoom_8_fine.png}
	\caption{Zoom con 8 scheduler, range 95000-100000 prodotti}
  	\label{fig:zoom_8_fine}
\end{figure}

Con un solo scheduler invece l'andamento con un processo risulta simile
a quello con più processi, ma il tempo impiegato con 8 scheduler risulta
in generale inferiore rispetto all'impiego di un solo scheduler, questo
perchè Erlang riesce a parallelizzare il numero di prodotti da effettuare
su più processi.

	