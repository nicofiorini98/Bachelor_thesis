% arara: pdflatex: { shell: yes }
% arara: biber 
% arara: pdflatex: { shell: yes }
% arara: pdflatex: { shell: yes }

\pdfminorversion=7  % see https://www.overleaf.com/learn/latex/%5Cpdfminorversion and https://www.overleaf.com/learn/latex/%5Cpdfinclusionerrorlevel
                    % and the answer in https://tex.stackexchange.com/questions/542343/epstopdf-pdfminorversion-7-error


\documentclass[a4paper, 12pt, oneside]{book}

%-----------------------------------------------------------------------------
% Packages
%-----------------------------------------------------------------------------
\usepackage[T1]{fontenc}            % Font encoding
\usepackage[utf8]{inputenc}         % Input encoding
\usepackage[english,italian]{babel} % Text Language (secondaria, primaria)
\usepackage[babel]{csquotes}        % Dipendenza di babel
\usepackage{microtype}              % Improves text writing
\usepackage[backend=biber,sorting=none]{biblatex}% Imports biblatex package
\usepackage{setspace}
\setcounter{biburlnumpenalty}{9000}
\setcounter{biburlucpenalty}{9000}
\setcounter{biburllcpenalty}{9000}
\addbibresource{./BackMatter/Tesi.bib}% Import the bibliography file
\usepackage{standalone}
\usepackage{graphicx}               % Required for inserting images
\graphicspath{ {./Images/} }        % Folder path to images. Path relative to the main .tex file  
\usepackage{subfig}
\usepackage{caption}
\captionsetup{format=hang,labelfont=bf}
\usepackage{float}                  % For figures
% \usepackage[x11names]{xcolor}
\usepackage{xcolor}
\usepackage{fancyhdr}               % For the headings      
\usepackage{listings}               % Uso dei listing per il codice
\usepackage{svg}
\usepackage{lipsum}                 % Dummy package
\usepackage{afterpage}
\usepackage[most]{tcolorbox}
% \tcbuselibrary{breakable}
\usepackage{tikz}
\usetikzlibrary{mindmap,shadows}
\usepackage{shapepar}
\usepackage{titlesec}
\usepackage{comment}
\usepackage[font={itshape,small}]{quoting}
\quotingsetup{font=small}
\usepackage{mathtools} % loads also amsmath package
\usepackage{amssymb}
\usepackage{amsfonts} % for \mathbb
\usepackage{amsthm}
\usepackage{bm}  % for accessing bold math symbols
\usepackage{cancel}
\usepackage{booktabs} % for tables
\usepackage{algorithm}
\usepackage{algpseudocode}  %load automatically the algorithmicx package
\usepackage{customization} % IMport custom preamble 
\usepackage[hidelinks]{hyperref}
% \usepackage{minted} % for language highlight
% \usemintedstyle{friendly}
\hypersetup{
    pdfstartview = XYZ null null 1.00,  % open the generated pdf with 100% zoom by default
    pdftitle={Tesi}, 
}


%-----------------------------------------------------------------------------
% Document Start
%-----------------------------------------------------------------------------
\begin{document}

\frontmatter
%-------------------------------------------------------------------------
%\begin{titlepage}

\thispagestyle{empty}

\begin{center}
    \Large{UNIVERSITÀ DEGLI STUDI DI \\ CASSINO E DEL LAZIO MERIDIONALE}
\end{center}

%---------------------
% Unicas_Emblem 
%---------------------
\begin{figure}[!htp]
    \centering
    \includegraphics[keepaspectratio=true,scale=0.25]{images/Frontespizio/Unicas_logo.pdf}
\end{figure}

\begin{center}
    \normalsize{DIPARTIMENTO DI INGEGNERIA ELETTRICA \\ E DELL'INFORMAZIONE “MAURIZIO SCARANO”}
    \vspace{4mm}
    \\ \normalsize{CORSO DI LAUREA IN INGEGNERIA INFORMATICA \\ E DELLE TELECOMUNICAZIONI}
\end{center}


\vspace{7mm}


\begin{center}
    \LARGE{\textbf{ Assessment delle performance di Elixir nell'ambito IOT.\\ }}
\end{center}


\vspace{20mm}


\begin{minipage}[t]{0.47\textwidth}
	{\normalsize{\textbf{Relatore:}}{\normalsize\vspace{3mm}
	\\ \large{Prof. Ciro D'Elia}}}
\end{minipage}
\hfill
\begin{minipage}[t]{0.47\textwidth}\raggedleft
	{\normalsize{\textbf{Candidato:}}{\normalsize\vspace{3mm} 
        \\ \large{Nico Fiorini}}}
\end{minipage}


\vspace{30mm}


{\centering{\large{ANNO ACCADEMICO 2022/2023}}\par}



%\end{titlepage}


\cleardoublepage
    \thispagestyle{empty}
        \vspace*{\stretch{1}}
            \begin{flushright}
				Dedicata ai miei nonni,\\
				a chi ci è sempre stato,\\
				a chi mi ha osservato e sostenuto.
				\\
				\vspace{2em}

				There's a lot of beauty in the ordinary things\\
				\textit{The Office}

            \end{flushright}
        \vspace{\stretch{2}}
\cleardoublepage
\afterpage{\null\thispagestyle{empty}\clearpage}
 
\begingroup 
\titleformat{\chapter}
{\normalfont\Huge\bfseries\centering} %shape
%{\centering} % format
{}% label
{} % sep
{}  

\chapter*{Abstract}

L' industria del software si trova a fronteggiare la 
necessità di sviluppare software sempre più scalabili 
e performanti per fronteggiare l'aumento degli utenti 
e di servizi che ne fanno utilizzo.
In questo contesto, Elixir, un linguaggio di programmazione 
funzionale e concorrente basato su Erlang, emerge come una 
scelta promettente per la costruzione di sistemi altamente 
affidabili e reattivi, nonché la sua capacità di
gestire carichi di lavoro intensivi e flussi di dati in
tempo reale tipici delle applicazioni IoT.

Questo studio si propone di analizzare le caratteristiche 
di Elixir che lo rendono un linguaggio degno di nota,
e attraverso una serie di esperimenti empirici si misurano le
performance di Elixir, anche mettendolo a confronto con altre
soluzioni più diffuse.

I risultati di questa ricerca forniscono una base per valutare
l'efficacia di Elixir, contribuendo così alla comprensione del
ruolo che questo linguaggio può svolgere nel futuro dello
sviluppo delle applicazioni IoT.

\endgroup

% s\include{}

\tableofcontents % Prints the main table of contents

%-------------------------------------------------------------------------
\mainmatter

% Introduzione
\chapter{Introduzione}

Elixir è un linguaggio di programmazione dinamico e funzionale
sviluppato nel 2012 da José Valim,
con l'obbiettivo di favorire uno sviluppo più agevole
sulla Erlang Virtual Machine(VM) e affrontare la sfida legata
a sistemi distribuiti, scalabili e affidabili mantenendo al
contempo la compatibilità con l'ecosistema di Erlang.
e produttività nella macchinia virtuale di Erlang.
Elixir si è affermato come una promettente scelta
nell'industria del software,specialmente in contesti dove è
richiesta scalabilità, tolleranza agli errori e reattività 
grazie al suo approccio concorrenziale.

Nell'ambito IoT 

In particolare, Elixir può risultare vantaggioso nel
campo dell'IoT per diversi motivi:
\begin{enumerate}
	\item \textbf{Concorrenza}: Nell'ambito dell'IoT, la gestione
	simultanea di dispositivi è essenziale.
	Elixir, grazie alla concorrenza basata su processi leggeri,
	e tramite il modello basato su attori può consentire il monitoraggio
	e il controllo efficiente di numerosi dispositivi contemporaneamente.
	\item \textbf{Fault Tolerance}: 
	Data la natura degli ambienti IoT, dove i dispositivi possono
	guastarsi improvvisamente, Elixir offre strumenti per la supervisione
	e la gestione degli errori, garantendo la continuità delle operazioni
	anche in caso di fallimenti.
	\item \textbf{Sviluppo Rapido e Manutenzione}:
	Elixir è un linguaggio moderno che offre una sintassi efficiente e snella,
	oltre a strumenti di sviluppo come Mix per la gestione delle dipendenze
	e l'ambiente interattivo iex. La presenza di un package manager (Hex)\cite{Hex15:online}
	e la possibilità di generare automaticamente la documentazione
	facilitano il processo di sviluppo e manutenzione del codice.
	
\end{enumerate}

\section{IoT}

L'Internet of Things o comunemente abbreviato IoT, rappresenta una
rete di dispotivi capaci di acquisire informazioni tramite sensori
o fare operazioni fisiche tramite attuatori.

Un architettura IoT ha una vasta gamma di applicazioni ed il suo uso 
sta crescendo velocemente nella società moderna, basta pensare alle
molteplici applicazioni che spaziano dall'uso industriale
per l'automazione, all'uso individuale con la domotica.

\subsection{Architettura IoT}

Un'architettura base per l'IoT può essere composta d quattro
livelli come in figura \ref{fig:architettura_iot}

\begin{figure}[!htp]
    \centering
    \includegraphics[keepaspectratio=true,scale=0.3]{images/architettura_iot.png}
	\caption{Architettura a 4 livelli}
  	\label{fig:architettura_iot}
\end{figure}

\begin{enumerate}
	\item Il livello Sensori è responsabile di collezionare dati da
	diverse sorgenti, i dispositivi fisici possono essere potenzialmente
	migliaia. Tutti questi dispositivi hanno il compito di
	inviare o ricevere dati ai livelli superiori.
	\item Il livello Network di un architettura IoT permette ai
	dispositivi di comunicare con il livello tre.
	Esempi di tecnologie di rete possono essere la rete WiFi,
	Zigbee e reti cellulari come il 5G.
	\item Il livello Data Processing gestisce la
	raccolta, l'analisi e l'interpretazione dei dati
	provenienti dai dispositivi IoT.
	Include tecnologie come sistemi di gestione dei dati,
	piattaforme di analisi e algoritmi di machine learning.
	\item Il livello delle applicazioni nell'architettura
	IoT è quello che interagisce direttamente con 
	l'utente finale, fornendo interfacce user-friendly 
	per controllare i dispositivi IoT. 
	Include app mobili, portali web e altre interfacce utente,
	oltre a servizi middleware per la comunicazione tra dispositivi.
	Inoltre, offre funzionalità di analisi e elaborazione dati,
	come algoritmi di machine learning e strumenti di visualizzazione.
\end{enumerate}

\subsubsection{Nerves}

Un framework open-source utilizzato per Elixir
è Nerves \cite{NervesPr90:online}, consente di
sviluppare software embedded basato su Erlang/OTP.

Offre agli sviluppatori un ambiente di sviluppo 
coerente e potente per la creazione di firmware
e software per dispositivi IoT e embedded.
Nerves semplifica il processo di sviluppo e distribuzione
di software per dispositivi embedded,
consentendo agli sviluppatori di creare facilmente
immagini firmware personalizzate e di integrare le funzionalità
richieste dai dispositivi.

Un esempio di utilizzo di Nerves nell'ambito IoT si ha
con l'azienda SparkMeter \cite{SparkMet65:online},
un azienda con l'obbiettivo di aumentare l'accesso all'elettricità
offrendo soluzioni di gestione della rete che 
consentono alle aziende di servizi pubblici nei mercati emergenti
di gestire sistemi finanziariamente sostenibili,
efficienti e affidabili.

Due dei loro prodotti sono contatori elettrici
intelligenti e software di gestione della rete.
Questi possono essere utilizzati per misurare
il consumo di elettricità, raccogliere informazioni
sulla salute di una rete elettrica e gestire
la fatturazione.

Una panoramica della loro architettura è mostrata
in figura \ref{fig:architettura_spark}, dove sono
mostrati degli Smart Meters, dei sistemi embedded
responsabili di collezionare misure del consumo
dell'elettricità. Comunicano l'un l'altro tramite
una rete mesh e comunicano con il Grid Edge
management Unit, un altro sistema Embedded che
riceve e processa dati da migliaia di smart meters.
Il Grid Edge managmement unit comunica con il cloud
che processa i dati così da poter essere mostrati
da una User interface \cite{Embedded35:online}.
Sia il Grid Edge Management Unit che il server
fanno uso di Elixir, infatti l'infrastruttura
dei sistemi Embedded non sono affidabili, 
la rete per la comunicazione con cui comunicano i
dispositivi sono via radio,
possono diventare irraggiungibili per vari fattori,
come una mancanza di elettricità, guasto o mancanza di rete.
Il sistema così ha bisogno di essere Fault-tolerant.
Inoltre utilizzando il meccanismo Port di Elixir,
vengono usati i vantaggi di Rust per processare
i dati dagli Smart Meters.

\begin{figure}[!htp]
    \centering
    \includegraphics[keepaspectratio=true,scale=0.33]{images/sparkmeter-new-architecture.png}
	\caption{Un esempio di architettura IoT di SparkMeter\cite{Embedded35:online}}
  	\label{fig:architettura_spark}
\end{figure}



\subsection{Introduzione allo studio svolto}

Il trattato esplora Elixir concentrandosi su due aspetti principali:
la semplicità e le performance. Si analizzano i punti di forza di 
un linguaggio funzionale e come questi sono sfruttati in Elixir, con 
un focus sulla concorrenza. Nella scelta di un linguaggio, la semplicità
è fondamentale e deve essere accessibile a tutti i programmatori.
Tuttavia, l'efficienza è altrettanto importante, quindi vengono condotti
test empirici per valutare le performance di Elixir.

In particolare il lavoro effettuato è così ripartito: 


\begin{itemize}
	\item Nel capitolo 2 si discute del linguaggio funzionale,
	esaminando le astrazioni offerte da Elixir per lo svilluppo
	di codice fault tolerant, si tratta la concorrenza e come la Erlang VM
	si occupa della gestione dei processi.
	\item Nel capitolo 3 si spiega il lavoro sperimentale svolto e i risultati
	ottenuti, in particolare vengono effettuati quattro test, i primi due
	per valutare l'efficienza della concorrenza della VM, il terzo per
	sperimentare l'interoperabilità con il codice C/C++ confrontando le
	performance del codice C/C++ con quello di Elixir, ed il quarto
	si occupa di misurare il throughput di un server Http scritto in
	Elixir confrontandolo con dei semplici server scritti in Node
	e Python.

\end{itemize}




% Capitolo 2
\chapter{Caratteristiche di Elixir}\label{chap:Caratteristiche}

\section{Introduzione}



In questo capitolo, esamineremo le caratteristiche distintive
di Elixir, un linguaggio di programmazione funzionale e
concorrente che sfrutta appieno la potenza della piattaforma
OTP (Open Telecom Platform). Non si vuole coprire ogni dettaglio
del linguaggio, ma mettere in evidenza le caratteristiche
fondamentali per iniziare a capire come pensare il codice
con questo linguaggio e quali sono le caratteristiche che
lo rendono un opzione da considerare in determinati contesti.

Elixir, scritto in Erlang ed eseguito sulla macchina virtuale
Erlang (BEAM), eredita gli obiettivi di Erlang, ma apporta
miglioramenti significativi per rendere il linguaggio più
appetibile e moderno.

Erlang, nato nel 1986, è stato progettato per semplificare
lo sviluppo di software concorrente robusto e scalabile.
Elixir si basa su queste fondamenta solide, offrendo un'API
più pulita e astrazioni dell'OTP che consentono ai programmatori
di ragionare a un livello più elevato, facilitando la scrittura
di codice concorrente in modo intuitivo.

Una delle massime principali di Erlang e, di conseguenza,
di Elixir, è "Let it crash" (Lascia che si schianti),
che riflette l'approccio alla gestione degli errori nei sistemi
concorrenti, incoraggiando la gestione degli errori tramite
il rilancio e la supervisione anziché il blocco del processo.

Per capire come lavorare con questo linguaggio, bisogna affrontare
un po' di questioni e farsi un po' di domande.
Bisogna capire come la macchina virtuale Beam affronta la concorrenza,
Elixir in particolare è un linguaggio orientato alla concorrenza e 
le astrazioni che fornisce sono proprio per far sì che
si programmi in modo concorrenziale portando ad avere un
codice responsivo. 

Gestendo in modo ottimale i processi attraverso 
il meccanismo di Supervision, il software diventa fault-tolerant.
Un altro punto che si affronta è l'immutabilità dei dati, un concetto
chiave in Elixir ed Erlang, ed è proprio questa caratteristica
che ci semplifica la programmazione concorrenziale.

\section{Il paradigma funzionale}
Come già accennato Elixir è un linguaggio funzionale, dove
il concetto di funzione ricopre il ruolo di protagonista,
i dati sono immutabili e il codice è dichiarativo.

Questo modo di vedere le cose deriva dal Lambda calcolo o $\lambda$-calcolo
\cite{Lambdaca75:online}
un sistema formale definito da Alonzo Church nel 1936, sviluppato
per definire formalmente le funzioni e il loro calcolo.

In un paradigma basato su stati come la programmazione ad oggetti
spesso si hanno variabili condivise mutabili, ovvero, più parti del codice
possono riferirsi alla stessa variabile, e questo complica la programmazione
multithreading dovendosi preoccupare di meccanismi come il blocco
sincronizzato o il locking per evitare le race condition tra più parti
del codice, e non è immediato scrivere del codice concorrenziale sicuro
e spesso si riscontrano comportamenti indeterminati.
In un paradigma funzionale si predilogono le variabili immutabili che
aggirano questo problema riducendo il rischio di scrivere codice
concorrenziale non sicuro.

Cambiare paradigma non è immediato, un paradigma si può dire che
definisce il modo di pensare al problema, nella programmazione ad
oggetti per esempio si definiscono le cosidette classi,
pensando al problema come oggetti che possono comportarsi in un
determinato modo attraverso le funzioni definite su di esso.
Perciò si pensa ad un oggetto che ha un comportamento e che cambia
il suo stato nel tempo, un modo di sviluppare intuitivo ma non sempre
ottimale per la risoluzione di problemi. Nella programmazione funzionale
si cambia prospettiva, ovvero si ha un input, si passa l'input 
alla funzione e si ottiene la trasformazione dell'input ottenendo
l'output.

In poche parole un linguaggio funzionale assume che scrivere un
software complesso sia più facile nel momento in cui il codice ha
queste proprietà:
\begin{itemize}
    \item I dati sono immutabili
    \item Le funzioni sono pure, ovvero, il risultato di una funzione
    dipende soltanto dai suoi parametri in input.
    \item Le funzioni non generano effetti oltre il suo valore restituito.
\end{itemize}

Con queste proprietà si ha più controllo del flusso del programma,
anche se non sempre possono essere soddisfatte.


\subsection{Struttura di un progetto Elixir}

Elixir è un linguaggio moderno, e come ogni linguaggio moderno che
si rispetti fornisce un tool per la creazione e configurazione di
progetti, questo tool si chiama \textbf{Mix}.

\subsubsection{Il tool Mix}
È possibile creare un progetto con il comando:

\begin{lstlisting}[language=none]
mix new <nome-progetto>
\end{lstlisting}

Verrà creata una struttura per il progetto, e nel progetto saranno
presenti una cartella \textbf{lib} dove andrà il codice,
una cartella \textbf{test} per creare degli Unit Test, ma soprattutto
viene creato un file \textbf{mix.exs} per la configurazione del progetto.

Come sappiamo Elixir fornisce anche un ambiente interattivo (\textbf{iex}) per
testare il nostro codice, ed è consentito avviare questo ambiente nel
dominio della nostra applicazione con il comando:

\begin{lstlisting}[language=none]
iex -S mix 
\end{lstlisting}
Si può compilare il progetto con:
\begin{lstlisting}[language=none]
mix compile
\end{lstlisting}

Con Mix possiamo includere e scaricare facilmente anche librerie esterne attraverso
il package manager.\cite{HexDocs99:online}

\subsubsection{Moduli}
Elixir organizza il codice in Moduli, che permettono di definire
le funzioni dentro dei namespace,permettendoci così di separare
le responsabilità raggruppando le funzioni.

Ci sono varie cose che si possono definire dentro un modulo,
si possono definire delle \textbf{struct}
ma cosa più importante si possono definire i cosiddetti 
\textbf{Behaviour}, un modo per definire un interfaccia Api,
Elixir fornisce delle astrazioni proprio attraverso questi.
%TODO menzionare GenServer e Supervisor
Ciò che si vuole evidenziare ora è che il progetto è definito
in moduli, ed è il modo che Elixir fornisce per organizzare il
codice.

% ----------------------------------------------------------------------------

\subsection{Basi dichiarative}

Come già accennato, Elixir adotta un approccio
dichiarativo nella definizione delle funzioni.
Questo si contrappone all'approccio imperativo,
che si concentra su "come posso risolvere questo problema?",
mentre quello dichiarativo si pone la domanda 
"come posso definire un problema?".

Nell'esempio \ref{lst:somma_n_cpp} è presentato un approccio
imperativo al problema "somma dei primi n elementi", mentre
nell'esempio \ref{lst:somma_n_elixir} è presentato l' approccio dichiarativo
con Elixir.

\renewcommand\lstlistingname{Esempio}

\begin{lstlisting}[language=cpp, caption={Somma N elementi},captionpos=b,
	label={lst:somma_n_cpp}]
int sum_first_n(n){
  int sum=0;
    for(int i=1;i++;i<=n){
      sum+=i;
    }
    return sum;
}
\end{lstlisting}

\begin{lstlisting}[language=elixir, caption={Somma N elementi},captionpos=b,
	label={lst:somma_n_elixir}]
defmodule Sum do
  def sum_recursive(0), do: 0
  def sum_recursive(n), do: n + sum_recursive(n - 1)
end
\end{lstlisting}

In un approccio imperativo bisogna controllare il flusso








% che d'altronde è un sistema equivalente alla
% Turing Machine
\section{Concorrenza}

La concorrenza è un concetto fondamentale nell'ambito
dello sviluppo software, si riferisce alla capacità
di eseguire più attività contemporaneamente.
Questo è particolarmente importante in applicazione
che devono gestire molteplici operazioni in parallelo
come ad esempio un server web, applicazioni di
elaborazione dati.

La concorrenza consente a un programma di sfruttare appieno
le risorse disponibili, aumentando l'efficienza e migliorando
le prestazioni complessive.
Inoltre, permette di creare sistemi più reattivi e
scalabili, in grado di gestire un numero crescente di
richieste senza compromettere le prestazioni.

In generale, la gestione della concorrenza può essere complessa
e soggetta a errori o comportamenti indeterminati, infatti,
più processi o thread possono interagire tra loro in modo
imprevedibile, essendo così costretti a usare tecniche di
locking in concomitanza di memoria condivisa trà più thread.

Nel contesto di Elixir, la concorrenza è un concetto centrale
e viene gestita attraverso il modello di programmazione
basato su processi leggeri, tutti isolati tra loro con il proprio
stack ed il proprio heap.

% ------------------------------------------------------

\subsection{La concorrenza in Beam}

La concorrenza gioca un ruolo chiave per un software
che vuole essere altamente responsivo.
La concorrenza è raggiunta nella piattaforma Erlang
con il concetto di Processo, per processo non si intende
il processo del sistema operativo, ma processo della
VM, chiamato processo e non thread in quanto non
condividono memoria tra di loro e sono completamente
isolati.

Un server tipico deve gestire migliaia di richieste, e
gestirle concorrenzialmente è essenziale per non far
rimanere in attesa il client. Quello che si vuole è
gestirli parallelamente il più possibile sfruttando
più risorse della Cpu disponibile.
Quello che fa la macchina virtuale per noi è permetterci
la scalabilità, più richieste allora più risorse da allocare.

Siccome il processo è isolato, un errore in una richiesta
ad esempio può essere localizzato senza avere impatto
sul resto del sistema, così creando anche un sistema robusto
agli errori.

In Beam due processi sono eseguiti concorrenzialmente e
se sono disponibili due Cpu core, forse sono eseguiti anche
in parallelo, la gestione della concorrenza è gestita
dalla macchina virtuale come in figura \ref{fig:concorrenza_beam}
in un computer dual core.


\begin{figure}[!htp]
    \centering
    \includegraphics[keepaspectratio=true,scale=0.25]{images/beam_concurrency.png}
	\caption{Concorrenza nella VM Beam \cite{elixirInAction5}}
  	\label{fig:concorrenza_beam}
\end{figure}

In Beam i processi sono entità leggere e sono gestiti
dalla macchina virtuale usando il proprio scheduler.
In dettaglio, la VM alloca di default uno scheduler per ogni
core logico disponibile, 


%--------------------------------------------------------------

\subsection{Concorrenza basata su attori}

Elixir usa un modello di concorrenza basato su attori,
Gli attori sono entità di elaborazione indipendenti
che eseguono operazioni in modo asincrono, questi attori
non sono altri che  processi che vengono identificati
attraverso un \textbf{PID} univoco.
Gli attori sono isolati l'uno dall'altro e
comunicano solo attraverso lo scambio di messaggi,
questo scambio avviene attraverso
dei canali di comunicazione detti \textbf{"mailbox"}.
Ogni processo ha una propria mailbox dove avviene la
ricezione del messaggio.

Conoscendo il PID di un processo può avvenire la comunicazione
attraverso la primitiva fornita dal linguaggio.

Un processo può essere creato tramite la primitiva \textbf{spawn/1},

\begin{lstlisting}[language=elixir, caption={Creazione processo},captionpos=b,
	label={lst:creazione_processo}]
iex(1)> pid = spawn(fn -> IO.puts("Hello") end)
Hello
#PID<0.114.0>
\end{lstlisting}
	\section{OTP Platform}

La concorrenza sembra gestita quasi come un gioco in Elixir,
ma ciò in genere non basta per semplificare lo sviluppo
di un software, per questo Elixir supporta l'insieme di
librerie OTP sviluppate per Erlang, con un Api più pulita
e moderna rispetto al suo predecessore.
L'OTP (Open Telecom Platform) è un insieme di librerie,
strumenti e linee guida per sviluppare dei sistemi
scalabili e resilienti in Erlang ed
Elixir, basti pensare che per via dell'immutabilità
dei dati e della natura funzionale del linguaggio,
non possiamo avere variabili globali che mantengono
uno stato, e per fare ciò, Elixir consente di dare
la responsabilità nel mantenimento di uno stato
ad un processo. Astrazioni come l'\textbf{Agent} o il \textbf{GenServer}
ci consentono di mantenere uno stato senza dover reinventare la
ruota nello scrivere un modulo soltanto per mantere un semplice stato.
L'approccio nella programmazione è totalmente differente
e piuttosto singolare rispetto ai più comuni linguaggi di programmazione,
ma è proprio questa singolarità che può portare dei vantaggi in alcune
nel mondo concorrenziale. 

%---------------------------------------------------------------------------------

\subsection{GenServer}
Si può volere più controllo rispetto alle primitive utilizzate
per gestire la concorrenza, un OTP server è un modulo con il
"Behaviour" GenServer. 
Il "Behaviour" è un meccanismo che consente di definire
uno schema comune per un tipo specifico di processo.

Ad esempio il GenServer Behaviour definisce le funzioni e
le interfacce necessarie per creare un processo server in grado
di gestire le richieste in modo asincrono.
Utilizzando il GenServer Behaviour, è possibile definire i comportamenti
di base del server e personalizzarli secondo le esigenze specifiche
dell'applicazione. 
Questo fornisce un alto livello di astrazione per la gestione
dei processi e semplifica lo sviluppo di sistemi concorrenti e
distribuiti in Elixir.

Il vantaggio di utilizzare un GenServer è che ha un'insieme
di interfacce standard e include funzionalità di tracciamento
e segnalazione degli errori. Si può anche mettere dentro un
albero di supervisione.

Questo Behaviour astrae l'interazione Client-server, come si può vedere
in figura \ref{fig:client_server}  \cite{GenServe6:online}.

\begin{figure}[!htp]
    \centering
    \includegraphics[keepaspectratio=true,scale=0.20]{images/GenServer.png}
	\caption{Interazione Client-Server}
  	\label{fig:client_server}
\end{figure}

Per implementare il behaviour GenServer, bisogna affidarsi alla
documentazione di GenServer, in particolare vanno ridefinite delle
callback, ed ogni funzione può restituire un determinato insieme di
strutture dati.

Nell'esempio \ref{lst:stackGenServer} viene implementata una struttura dati
per mantere uno Stack di dati \cite{GenServe6:online}.

\begin{lstlisting}[language=elixir, caption={Implementazione Stack},captionpos=b,
	label={lst:stackGenServer}]
defmodule Stack do
use GenServer

# Client

def start_link(default) when is_binary(default) do
  GenServer.start_link(__MODULE__, default)
end

def push(pid, element) do
  GenServer.cast(pid, {:push, element})
end

def pop(pid) do
  GenServer.call(pid, :pop)
end

# Server (callbacks)

@impl true
def init(elements) do
  initial_state = String.split(elements, ",", trim: true)
  {:ok, initial_state}
end

@impl true
def handle_call(:pop, _from, state) do
  [to_caller | new_state] = state
  {:reply, to_caller, new_state}
end

@impl true
def handle_cast({:push, element}, state) do
  new_state = [element | state]
  {:noreply, new_state}
end
  end
\end{lstlisting}

Nell'esempio possiamo vedere che il modulo stack implementa
la funzione \textbf{init/1} che inizializza lo stato con gli elementi
iniziali quando il server viene avviato, la funzione \textbf{handle\_call/3}
chiamata per le operazioni di :pop dello stack, in particolare
è una funzione sincrona, quindi viene usata quando ci si aspetta
un valore di ritorno, infatti restituisce il valore di
testa dello stack. La funzione \textbf{handle\_cast/2} invece viene usata per
le operazioni asincrone, quindi nel caso in esame per l'operazione di
push nello Stack.

Quindi il GenServer è un'astrazione che:

\begin{itemize}
	\item Incapsula un servizio condiviso.
	\item Mantiene uno stato.
	\item Permette un'astrazione concorrente ad un servizio condiviso \cite{adoptingElixirchap5pag96}.
\end{itemize}

%---------------------------------------------------------------------------------

\subsection{Supervisor}

Un Supervisor è un modulo che implementa il Behaviour Supervisor,
sono processi specializzati per un solo scopo: monitorare altri processi
che chiameremo d'ora in poi processi figli.

Sono questi Supervisor che ci semplificano lo sviluppo di applicazioni
fault-tolerant, supervisionando i processi figli.
Bisogna quindi decidere quali sono i processi figli da supervisionare,
e una volta avviato il Supervisor con la funzione \textbf{start\_link/2},
questo ha bisogno di sapere come avviare, fermare o riavviare i suoi figli
in caso di errore o uscita imprevista.

Per questo i figli hanno bisogno di avere una funziona \textbf{child\_spec/1}
che definisce il comportamento del supervisore. I moduli che implementano
lo GenServer, oppure un Agent automaticamente definiscono questa funzione,
quindi non c'è bisogno di modificare il modulo.
La funzione child\_spec/1 è una funzione che restituisce una Map per
configurare il comportamento in caso di supervisione.


% Capitolo 3
\chapter{Performance - Test sperimentali}

% \newpage
\section{Introduzione}
In questo capitolo si vuole capire quali potenzialità
la piattaforma di Erlang/Elixir porta, in modo da poter
capire quali siano le applicazioni possibili con questa tecnologia.

Vogliamo capire i limiti di questa tecnologia, attraverso una serie
di test empirici e provare a fare qualche confronto con altri
linguaggi.

% ---------------------Hardware utilizzato-------------------------- 
\subsection{Hardware utilizzato}

I test vengono eseguiti sul sistema operativo linux, viene riportato in
figura \ref{fig:neofetch} l'output del tool \textbf{neofetch} che riporta
le caratteristiche del computer utilizzato per l'esecuzione dei test,
in particolare notiamo che la CPU di riferimento dei test
ha 8 core logici e 4 fisici, questa caratteristica è importante
in quanto ci aspettiamo che i test concorrenti diano il meglio di
sè con più di 4 processi attivi.

\begin{figure}[!htp]
    \centering
    \includegraphics[keepaspectratio=true,scale=0.3]{images/neofetch.png}
	\caption{Output neofetch}
  	\label{fig:neofetch}
\end{figure}

% ---------------------Test Interoperabilità-------------------------- 



\section{Test concorrenziale}

In questo Test empirico si vuole vedere come Elixir
si comporta all'aumentare dei processi, facendo una
serie di prodotti, 
\section{Test concorrenziale con IO}
\section{Test interoperabilità con C/C++}

\renewcommand\lstlistingname{Listing}

In questo test empirico viene utilizzata la libreria Benchee
che ci permette di valutare le performance di esecuzione di
funzioni, in particolare le funzioni da eseguire saranno due funzioni
scritte in Elixir, ed altre due con il meccanismo di integrazione
di cui si è parlato nel paragrafo \ref{subsec:mysubsection}
Il problema che si vuole risolvere è l'operazione somma
riportata nell'equazione \ref{eq:problem_interoperability}.
\begin{equation}
	result = \sum_{i=1}^{n}i \label{eq:problem_interoperability}
\end{equation}

Le funzioni implementate prendono come input una lista
di interi, e si vuole restituire una lista che contiene
i risultati dell'equazione \ref{eq:problem_interoperability}.

L'input di riferimento è: [ 1000000, 2000000, 5000000], non si
sono scelti numeri troppo piccoli per avere un gap visibile
tra le strategia di implementazione adottate.

Le funzioni si occupano di calcolare il risultato della somma
\ref{eq:problem_interoperability} per ogni elemento e
restituirà una lista con il relativo risultato per ogni elemento.

Le funzioni sono implementate con 4 strategie differenti:
\begin{enumerate}
	\item list\_sum\_recursive(): Funzione implementata in elixir
	tramite ricorsione.
	\item list\_sum\_recursive\_tail(): Funzione implementata in elixir
	tramite ricorsione ottimizzata, dove l'ultima istruzione è la
	chiamata ricorsiva, in modo che elixir ottimizza la ricorsione
	con la tail optimization.
	\item list\_sum\_port(): Funzione implementata in C++, Elixir interagisce
	con il codice tramite il meccanismo IPC Port.
	\item list\_sum\_iterative\_nif(): Funzione implementata in C++, Elixir
	interagisce con il codice C++ tramite il meccanismo NIF.
\end{enumerate}

Tutte le funzioni elixir che seguiranno in questo paragrafo
vengono messe dentro il modulo SpeedSum
come nel Listing \ref{lst:speedSum}:

\begin{lstlisting}[language=elixir,captionpos=b,
	caption={Modulo di riferimento},
	label={lst:speedSum}]
defmodule Speedsum do

# Funzioni implementate


end


\end{lstlisting}

% --------------------------------------------

\subsection{Funzione ricorsiva non ottimizzata}
Nel Listing \ref{lst:listsumrecursive} è riportata l'implementazione
della funzione ricorsiva che risolve il problema definito sopra.

\begin{lstlisting}[language=elixir,captionpos=b,
	caption={Funzione list\_sum\_recursive()},
	label={lst:listsumrecursive}]
def list_sum_recursive(list) when is_list(list) do
  Enum.map(list, fn x -> sum_recursive(x) end)
end

#caso base
defp sum_recursive(0), do: 0

defp sum_recursive(n), do: n + sum_recursive(n - 1)
\end{lstlisting}

%---------------------------------------------------------------------
\subsection{Funzione ricorsiva non ottimizzata}
Nel Listing \ref{lst:listsumrecursivetail} è riportata l'implementazione
della funzione ricorsiva ottimizzata con il metodo tail optimization,
Elixir come vedremo dai risultati del Benchmark risolve il problema
definito in modo molto più veloce, come si può vedere dal codice
implementato, per far sì che il passo ricorsivo sia l'ultima
istruzione della funzione, necessita di passare lo stato della
somma come parametro della funzione.

\begin{lstlisting}[language=elixir,captionpos=b,
	caption={Funzione list\_sum\_recursive\_tail()},
	label={lst:listsumrecursivetail}]
# Funzione di somma ricorsiva ottimizzata
def list_sum_recursive_tail(list) when is_list(list) do
  Enum.map(list, fn x -> sum_recursive_tail(x, 0) end)
end
  
#caso base
defp sum_recursive_tail(0, acc), do: acc

defp sum_recursive_tail(n, acc) when n > 0 do
  sum_recursive_tail(n - 1, acc + n)
end
\end{lstlisting}

%---------------------------------------------------------------------

\subsection{Funzione implementata in C++ tramite Port}

Con l'Inter process communication in Erlang ed Elixir, si può comunicare
con un processo esterno tramite il meccanismo Port, e si può comunicare con
i cosiddetti file descriptor, nel nostro caso è stata implementata la
comunicazione usando lo standard input e lo standard output. 
Per la comunicazione si deve scegliere un formato di codifica e decodifica
per la comunicazione, per non complicare le cose si è scelto
una comunicazione line by line, dove la lista viene inviata
da Elixir nel formato "1000000,2000000,5000000" scegliendo come
carattere delimitatore la ",".

Il codice C++ che risolve il nostro problema è riportato nel Listings
\ref{lst:list_sum_port}

\begin{lstlisting}[language=cpp,captionpos=b,
	caption={Funzione list\_sum\_port()},
	label={lst:list_sum_port}]
#include <algorithm>
#include <iostream>
#include <sstream>
#include <string>
#include <vector>
	
//Funzione che prende in input una stringa
//di interi separati da un delimitatore e
//restituisce un vector di interi
std::vector<long int> tokenizeToIntVector(const std::string &str,
                                          char delimiter) {
  std::vector<long int> tokens;
  std::stringstream ss(str);
  std::string token;
  while (std::getline(ss, token, delimiter)) {
    tokens.push_back(
      std::stol(token));
	}
	return tokens;
}
	
int main() {
  std::string line;
  // Ascolta lo standard input line by line
  while (std::getline(std::cin, line)) {

    char delimiter = ',';

    std::vector<long> numbers = tokenizeToIntVector(line, delimiter);
    std::vector<long> results;

    for (auto &n : numbers) {
      long sum = 0;
      for (int i = 0; i <= n; i++) {
        sum += i;
      }
      results.push_back(sum);
    }

    // Stampa sullo standard output una stringa
    // nello stesso formato di ricezione
    for (int i = 0; i < results.size() - 1; i++) {
      std::cout << results[i] << ",";
    }
    std::cout << results[results.size() - 1] << std::endl;
  }
  return 0;
}

\end{lstlisting}

Si può compilare il programma C++ tramite il compilatore gcc:
\begin{lstlisting}[language=none]
g++ -o list_sum_port <source-directory>/list_sum_port.cpp
\end{lstlisting}

Nel Listing \ref{lst:listsumportelixir} è riportata la funzione
che comunica con il programma C++ scritto per la risoluzione
del problema definito.

\begin{lstlisting}[language=elixir,captionpos=b,
	caption={Funzione list\_sum\_port()},
	label={lst:listsumportelixir}]

def list_sum_port(list) when is_list(list) do

  # Apertura del Port
  port = Port.open({:spawn_executable, "./priv/list_sum_port"},
                   [:binary,:use_stdio])

  # encoding del messaggio da inviare al port
  message = Enum.join(list,", ")

  # invio del messaggio al port
  Port.command(port, "#{message}\n")

  receive do
    {^port, {:data, result}} ->
      String.trim(result)

    {^port, {:exit_status, status}} ->
      IO.puts("Processo port terminato con codice di uscita #{status}")
  after
    1000 ->
      IO.puts("Timeout")
  end
  Port.close(port)
end
\end{lstlisting}

%-------------------- Funzione NIF -----------------------
\subsection{Funzione implementata in C++ tramite NIF}

Nel caso del metodo NIF non c'è bisogno di decidere un
metodo di codifica e decodifica, ma bisogna affidarsi
all'interfaccia che Erlang ci mette a disposizione per
far interfacciare il codice C con Elixir, l'interfaccia
supporta solo il C, quindi per poter utilizzare codice C++
si devono fare le dovute conversioni.
Nel Listing \ref{lst:list_sum_nif} è riportato il codice
NIF per la risoluzione del problema definito.


%// gcc -fPIC -shared -o elixir_nif.so elixir_nif.c -I $ERL_ROOT/usr/include/
\begin{lstlisting}[language=cpp,captionpos=b,
	caption={Funzione NIF},
	label={lst:list_sum_nif}]
#include "erl_nif.h" 
#include "string.h"
#include <vector>
	
	
// funzione per riempire un vector da una lista di Elixir
inline bool fillVector(ErlNifEnv* env,
                      ERL_NIF_TERM listTerm,
                      std::vector<long int>& result) 
{
  unsigned int length = 0;
  if (!enif_get_list_length(env, listTerm, &length)) {
    return false;
  }
	
  long int actualHead; 
  ERL_NIF_TERM head;
  ERL_NIF_TERM tail;
  ERL_NIF_TERM currentList = listTerm;
	
  // O(n), scorre tutta la lista
  for (unsigned int i = 0; i < length; ++i) {
    if (!enif_get_list_cell(env, currentList, &head, &tail)) {
      return false;
    }
    currentList = tail;
    if (!enif_get_long(env, head, &actualHead)) {
      return false;
    }
    result.push_back(actualHead);
  }
  return true;
  }
	
static ERL_NIF_TERM list_sum_iterative_nif(ErlNifEnv* env,
                                          int argc,
                                          const ERL_NIF_TERM argv[]) {
	
  std::vector<long int> a;
	
  if (!fillVector(env, argv[0], a)) {
    return enif_make_badarg(env);
  }
	
  //creazione dell'array per i risultati da restituire
  ERL_NIF_TERM results[a.size()];
	
  for(int i=0; i < a.size() ;++i){
    long int  sum = 0;
    for (int j = 1; j <= a[i]; j++) {
    sum += j;
  }

  ERL_NIF_TERM  temp = enif_make_long(env, sum);
  results[i] = temp;
  }

  ERL_NIF_TERM list = enif_make_list_from_array(env, results , a.size());
	
  return list;
}
	
// Definizione delle funzioni NIF
static ErlNifFunc nif_funcs[] = {
  {"list_sum_iterative_nif", 1, list_sum_iterative_nif}
};

ERL_NIF_INIT(Elixir.SpeedSum, nif_funcs, NULL, NULL, NULL, NULL)
	
\end{lstlisting}

Come si può vedere il NIF implementato è composto da una funzione
ausiliaria che rserve per riempire il vettore dalla lista di Elixir,
dall'implementazione della funzione NIF list\_sum\_iterative\_nif(),
e dalla macro ERL\_NIF\_INIT.

Si può notare che tutti i tipi che devono essere letti
dalla VM devono essere degli ERL\_NIF\_TERM, che possono essere
letti e scritti utilizzando l'API messa a disposizione dall'installazione
di Erlang, che si trova nel file "erl\_nif.h".

Per utilizzare la funzione implementata, il codice deve essere compilato
come una libreria condivisa, con il compilatore gcc si può compilare
nel seguente modo:

\begin{lstlisting}[language=none]
g++ -fPIC -shared -o list_sum_iterative_nif.so \\
    list_sum_iterartive_nif.cpp -I $ERL_ROOT/usr/include/
\end{lstlisting}

La directory \$ERL\_ROOT dipende dall'installazione di Erlang,
per saperne il valore si può eseguire:
\begin{lstlisting}[language=none]
elixir -e "IO.puts :code.root_dir()"
\end{lstlisting}

Per eseguire la funzione esterna implementata, basta definire
nel modulo di riferimento il seguente codice:

\begin{lstlisting}[language=elixir,captionpos=b,
	caption={Funzione list\_sum\_port()},
	label={lst:listsumportelixir}]

def load_nif do
  :ok = :erlang.load_nif(
	    String.to_charlist("priv/list_sum_iterative_nif"),
		0)
end

def list_sum_iterative_nif(_n) do
  :erlang.nif_error("Errore nel caricamento nif")
end
\end{lstlisting}



%--------------------- ESECUZIONE ------------------------
\subsection{Esecuzione Test}

Per l'esecuzione del test come precedentemente detto
ci si è affidati alla libreria Benchee \cite{bencheeo54:online}.

Per includere la libreria Benchee, basta inserire nella funzione
deps del file \textbf{mix.exs} del progetto Elixir il nome
della libreria con la versione, la funzione diventa
quella del Listing \ref{lst:benchee_dependencies}.


\begin{lstlisting}[language=elixir,captionpos=b,
	caption={Dipendenze di Benchee},
	label={lst:benchee_dependencies}]
defp deps do
  [
  {:benchee, "~> 1.0", only: :dev},     
  {:benchee_html, "~> 1.0", only: :dev},
  ]
end
\end{lstlisting}

Viene inserito anche il plug-in della libreria
per visualizzare il report in formato html per
una visione più chiara del risultato.

Con la libreria Benchee sono state testate tutte e quattro le
strategie di implementazione usate, ed è stato aggiunta la strategia
di formattazione HTML per visualizzare i report, nel Listing \ref{lst:speedtest_function}.
 

\begin{lstlisting}[language=elixir,captionpos=b,
	caption={Funzione speed\_test()},
	label={lst:speedtest_function}]
def speed_test() do
  list_input = [1_000_000, 2_000_000,5_000_000]
  Benchee.run(
  %{
   "sum_recursive" => fn -> list_sum_recursive(list_input) end,
   "sum_recursive_tail" => fn -> list_sum_recursive_tail(list_input) end,
   "sum_iterative_nif" => fn -> list_sum_iterative_nif(list_input) end,
   "list_sum_port" => fn -> list_sum_port(list_input) end
   },
   warmup: 4,
   memory_time: 4,
   formatters: [
     Benchee.Formatters.HTML,
     Benchee.Formatters.Console
  ])
end
\end{lstlisting}

La funzione costruisce un report in HTML che riporta la Tabella
in figura

\begin{figure}[!htp]
    \centering
    \includegraphics[keepaspectratio=true,scale=0.21]{images/tab_report.png}
	\caption{Report di Benchee}
  	\label{fig:report_tab_interoperabilita}
\end{figure}

Come possiamo notare le funzioni scritte in C++ risultano più veloci, nonostante
potrebbero essere ottimizzate ulteriormente. La strategia con il Port risulta
essere seconda alla strategia con NIF, come era previsto, e come
già detto nella documentazione di Erlang, il metodo con Port è più difficile
da utilizzare a primo impatto in quanto si deve decidere un protocollo di
codifica e decodifica, ma una volta scelto risulta il metodo più stabile
e affidabile 



\section{Test Http Server}
Nel seguente benchmark si vuole mettere a confronto un
server http scritto in Elixir utilizzando la libreria \textbf{Plug}\cite{Plug—Plu62:online}
con i server scritti in Python e Node, usando
la libreria \textbf{Flask}\cite{Welcomet46:online} per
Python e la libreria \textbf{Express}\cite{ExpressN36:online} per Node.
I benchmark vengono eseguiti sulla stessa macchina con il tool
\textbf{wrk}, wrk è uno strumento moderno di benchmarking HTTP
in grado di generare un carico significativo se eseguito su una
CPU multi-core \cite{wgwrkMod56:online}.

% \textbf{ab - Apache HTTP server benchmarking tool}
% \cite{abApache17:online}.

\subsection{Implementazione dei Server}

Gli http server implementati sono molto semplici, per
fare load testing si dovrebbero simulare situazioni
di calcolo reale sul server http, in questo caso si
vuole solo avere un idea di come Elixir performa
su una risposta breve ad una chiamata GET
all'indirizzo localhost:5000/ping, rispondendo
con un semplice html con la stringa "pong".

\subsubsection{Http Server Elixir}

Per utilizzare la libreria Plug basta inserire la dipendeza
richiesta nel file mix.exs:

\begin{lstlisting}[language=none, caption={Implementazione server con Plug},captionpos=b,
	label={lst:server_elixir}]
{:plug_cowboy, "~> 2.0"},             # http server library
\end{lstlisting}

Basta eseguire il comando "mix deps.get" per
scaricare le dipendenze.


Nel Listing \ref{lst:server_elixir} è riportato il codice che
avvia un http server con la libreria Plug.


\begin{lstlisting}[language=elixir, caption={Implementazione server con Plug},captionpos=b,
	label={lst:server_elixir}]
defmodule MyRouter do
  use Plug.Router
  
  plug :match
  plug :dispatch
  
  get "/ping" do
    send_resp(conn, 200, "pong")
  end
  
  match _ do
    send_resp(conn, 404, "oops")
  end
end
\end{lstlisting}

Per l'avvio del server ci si affida ad un Supervisor
riportato nel Listing \ref{lst:supervisor_http}.

\begin{lstlisting}[language=elixir, caption={Supervisor dell'applicazione HTTP},
	captionpos=b,label={lst:supervisor_http}]
defmodule InteroperabilityTest.MyHttpApplication do
  use Application
  require Logger

  def start(_type, _args) do

    children = [
      {Plug.Cowboy, scheme: :http,
	  plug: MyRouter, options: [port: cowboy_port()]}
    ]

    # opzioni per il supervisor del modulo Myhttp
    opts = [strategy: :one_for_one, name: MyHttpServer.Supervisor]

    Logger.info("Starting application on port #{cowboy_port()}...")
    Supervisor.start_link(children, opts)
  end

  defp cowboy_port, do: Application.get_env(:example, :cowboy_port, 3000)

end
\end{lstlisting}

%--------------------------------------------------
\subsubsection{HTTP Server Python}

Un semplice Server python come quello di Elixir si può fare con
la libreria Flask, il codice è riportato nel Listing \ref{lst:server_python}

\begin{lstlisting}[language=ipython, caption={Server Python(Flask)},
	captionpos=b,label={lst:server_python}]
from flask import Flask

app = Flask(__name__)
	
@app.route('/ping')
def ping():
    return 'pong'
	
if __name__ == '__main__':
    app.run(debug=False)

\end{lstlisting}

Assumendo di avere l'interprete Python
installato sulla macchina, per
avviare il server basta installare la libreria Flask:

\begin{lstlisting}[language=none]
> pip install flask
...
> python server.py
\end{lstlisting}


\subsubsection{HTTP Server Node}

Con Node il Server viene implementato con la libreria Express.

\begin{lstlisting}[language=ipython, caption={Server Node(Express)},
	captionpos=b,label={lst:server_node}]
const express = require('express');
const app = express();
	
app.get('/ping', (req, res) => {
	res.send('pong');
});
	
const PORT = process.env.PORT || 5000; 
app.listen(PORT, () => {
	console.log('Server is running on port ${PORT}');
});
	
\end{lstlisting}

Assumendo di avere Node.js installato sulla macchina
per avviare il server basta installare la libreria Express:
\begin{lstlisting}[language=none]
> npm install express
...
> node server.js
\end{lstlisting}

\subsection{Esecuzione Test Http}

Come accennato viene usato il tool wrk per eseguire
più richieste concorrenziali e multithreading, vengono
eseguite per semplicità sulla stessa macchina, sviluppi
futuri possono prevedere un Load Testing più complesso
con il tool Apache Jmeter, eseguendo i test su
macchine differenti simulando applicazioni di vita reale.
Come detto in questo luogo si vuole
avere la percezione di quanto Elixir sia adatto a questo
scopo, infatti l'architettura leggera di Elixir e il sistema
di gestione della concorrenza consente di gestire
migliaia di connessioni simultanee in modo efficiente
senza bloccare il processo principale. Ciò consente
di fornire tempi di risposta minimi, rendendo Elixir
una scelta ideale per applicazioni real-time.

Per installare wrk su un sistema debian, basta eseguire
\begin{lstlisting}[language=none]
> sudo apt-get update
> sudo apt-get install wrk
\end{lstlisting}

Una volta installato basta eseguire un server per volta
ed eseguire il comando per simulare richieste simutanee:

\begin{lstlisting}[language=none]
> wrk -t8 -c<n> -d10s --latency http://localhost:5000/ping
\end{lstlisting}

Il comando viene eseguito con 8 thread, e vengono lanciate
$n$ richieste concorrenziali simulando $n$ richieste
simultanee di utenti. L'opzione --latency serve per stampare la
distribuzione di latenza delle richieste.

\newpage
Un esempio di output:

\begin{lstlisting}[language=none]
> wrk -t8 -c100 -d10s --latency http://localhost:5000/ping
Running 10s test @ http://localhost:5000/ping
  8 threads and 100 connections
  Thread Stats   Avg      Stdev     Max   +/- Stdev
    Latency     1.66ms    1.33ms  34.83ms   83.83%
    Req/Sec     8.01k   626.33    10.90k    70.50%
  Latency Distribution
     50%    1.29ms
     75%    2.01ms
     90%    3.37ms
     99%    6.30ms
  639802 requests in 10.04s, 89.70MB read
Requests/sec:  63697.09
Transfer/sec:      8.93MB
\end{lstlisting}

Sono stati effettuati dei test al variare di $n$
richieste concorrenziali e sono riportati i risultati
ottenuti nelle Tabelle \ref{tab:elixir_report},
\ref{tab:node_report}, \ref{tab:python_report}.
Si può notare come Elixir performa in modo significativo
rispetto agli altri due server, e da notare anche la latenza media
di Elixir rispetto agli altri, d'altronde Elixir parallelizza
le richieste il più possibile, durante i test con il tool
\textbf{htop} si è riscontrato un uso della cpu prossima al 100\%
mentre con Node, il server è single Thread, perciò le richieste
vengono eseguite su un'unica CPU, e non si è percepito un'aumento
significativo delle CPU. E' anche da notare che anche il client
eseguito sulla stessa macchina è multithreading, e fa un uso
elevato della CPU anch'esso, perciò non si nega che Elixir
possa performare ancor di più.
Le righe con valori Null, sono test eseguiti in
cui il client chiude la connessione al client per via di un timeout.
Non si può affermare che Node e Python non possono
fare di meglio, i server andrebbero configurati meglio,
in questo luogo si è voluto vedere come Elixir con la
concorrenza leggera riesce a performare e parallelizzare
in modo ottimale.
Ci sono benchmark in rete in cui Elixir tramite il
framework \textbf{Phoenix},
riesce a gestire fino a 2 milioni di connessioni
WebSocket simultanee
su una singola macchina \cite{TheRoadt94:online}.

\begin{table}%[htbp]
  \centering
  \begin{tabular}{cccc}
    \toprule
    n & richieste/s & Latenza AVG & Stdev \\
    \midrule
    10 & 40943 & 334.87 \textmu s & 2.64 ms \\
    100 & 63697 & 1.66 ms& 1.33 ms \\
    1000 & 594442 & 42.68 ms & 157.75 ms\\
    10000 & 51872 & 81.47 ms& 24.58 ms\\
    20000 & 18092 & 215.25 ms & 78.27 ms\\
    \bottomrule
  \end{tabular}
  \caption{Elixir http server benchmark}
  \label{tab:elixir_report}
\end{table}

\begin{table}%[htbp]
  \centering
  \begin{tabular}{cccc}
    \toprule
    n & richieste/s & Latenza AVG & Stdev \\
    \midrule
    10 & 6463 & 1.79ms \textmu s & 6.61 ms \\
    100 & 6258& 15.32 ms& 2.33 ms \\
    1000 & 5621.09 & 171.61 ms & 21.95 ms\\
    10000 & 4988.81 & 519.53 ms& 70.12 ms\\
    Null & Null & Null & Null\\
    \bottomrule
  \end{tabular}
  \caption{Node http server benchmark}
  \label{tab:node_report}
\end{table}

\begin{table}%[htbp]
  \centering
  \begin{tabular}{cccc}
    \toprule
    n & richieste/s & Latenza AVG & Stdev \\
    \midrule
    10 & 1679 & 4.69 ms & 458.13 \textmu s\\
    100 & 1695 & 56.36 ms & 2.53 ms \\
    1000 & 1609 & 88.96 ms & 67.44 ms\\
    Null & Null & Null & Null\\
    Null & Null & Null & Null\\
    \bottomrule
  \end{tabular}
  \caption{Python http server benchmark}
  \label{tab:python_report}
\end{table}




% Capitolo 4
\chapter*{Conclusioni e sviluppi futuri}
\addcontentsline{toc}{chapter}{Conclusioni e sviluppi futuri}

\setlength{\parindent}{0pt}

Elixir si è dimostrato un linguaggio potente che
migliora molti aspetti dello sviluppo software,
soprattutto quello in ambito concorrenzile, sviluppare
è stato piacevole dimostrando di essere un linguaggio
moderno degno di nota.
\vspace{1.5em}

Ha dimostrato di avere uno scheduling molto
veloce che non degrada significamente le performance.
Se si necessita di computazioni dove Elixir non
ottiene prestazioni ottimali 
si può integrare con 
soluzioni più adatte come il C/C++, Rust o Python, tramite
inter process communication o tramite funzioni native NIF.
Ha dimostrato di avere una bassa latenza ottimale per
sistemi real time, ed è quello il suo principale utilizzo
adottato finora, ma ha dimostrato di essere una scelta promettente
anche nell'ambito dell'IoT, dove affidabilità reattività 
e concorrenza sono dei fattori chiave.
\vspace{1.5em}

Futuri studi possono vedere l'integrazione di Elixir
in software IoT, tramite il framework Nerves o con
soluzioni ad Hoc per il caso specifico.




%-------------------------------------------------------------------------
\backmatter

\printbibliography[heading=bibintoc,title={Bibliografia e Sitografia}] % Prints bibliography and display bibliography to toc
\begingroup 
\titleformat{\chapter}
{\normalfont\Huge\bfseries\centering} %shape
{}% label
{} % sep
{}  

\chapter*{Ringraziamenti}


\thispagestyle{empty}


\setlength{\parindent}{0pt}

{\large

\begin{onehalfspace}

La scelta di un obbiettivo difficile è stato un caso,
l'aiuto nel raggiungerlo non lo è stato. 

\vspace{1.5em}
Un doveroso e sentito ringraziamento al Prof. Ciro D'Elia,
che mi ha fornito preziosi spunti per questo studio.

\vspace{1.5em}
Un ringraziamento speciale va mia madre e mio padre
che mi hanno permesso di intraprendere questo percorso
con tutti i sacrifici che hanno fatto e continuano a fare,
permettendomi di compiere i miei passi con più leggerezza.

\vspace{1.5em}
Un altro ringraziamento speciale a Valentina che negli
ultimi due anni mi ha sostenuto aggiungendo
un colore alla mia vita in scala di grigi.

\vspace{1.5em}
Grazie a tutta la mia famiglia che mi ha supportato.

\vspace{1.5em}
Grazie a tutti i miei amici e i momenti
condivisi insieme rendendo la vita
più curiosa e divertente.


\vspace{1.5em}
Grazie.


\end{onehalfspace}
}

\endgroup


% N:B IMPORTANT!!!
%\mainmatter gives you normal chapter numbering and arabic page numbering.
%\frontmatter gives you UNNUMBERED chapters (so chapter* is redundant) and roman page numbering.
%\backmatter gives you UNNUMBERED chapters (so chapter* is redundant) and arabic page numbering.


\end{document}